%%%%%%%%%%%%%%%%%%%%%%%%%%%%%%%%%%%%%%%%%
% The Legrand Orange Book
% LaTeX Template
% Version 2.4 (26/09/2018)
%
% This template was downloaded from:
% http://www.LaTeXTemplates.com
%
% Original author:
% Mathias Legrand (legrand.mathias@gmail.com) with modifications by:
% Vel (vel@latextemplates.com)
%
% License:
% CC BY-NC-SA 3.0 (http://creativecommons.org/licenses/by-nc-sa/3.0/)
%
% Compiling this template:
% This template uses biber for its bibliography and makeindex for its index.
% When you first open the template, compile it from the command line with the 
% commands below to make sure your LaTeX distribution is configured correctly:
%
% 1) pdflatex main
% 2) makeindex main.idx -s StyleInd.ist
% 3) biber main
% 4) pdflatex main x 2
%
% After this, when you wish to update the bibliography/index use the appropriate
% command above and make sure to compile with pdflatex several times 
% afterwards to propagate your changes to the document.
%
% This template also uses a number of packages which may need to be
% updated to the newest versions for the template to compile. It is strongly
% recommended you update your LaTeX distribution if you have any
% compilation errors.
%
% Important note:
% Chapter heading images should have a 2:1 width:height ratio,
% e.g. 920px width and 460px height.
%
%%%%%%%%%%%%%%%%%%%%%%%%%%%%%%%%%%%%%%%%%

%----------------------------------------------------------------------------------------
%	PACKAGES AND OTHER DOCUMENT CONFIGURATIONS
%----------------------------------------------------------------------------------------

\documentclass[11pt,fleqn]{book} % Default font size and left-justified equations

\input{texs/structure.tex} % Insert the commands.tex file which contains the majority of the structure behind the template

%\hypersetup{pdftitle={Title},pdfauthor={Author}} % Uncomment and fill out to include PDF metadata for the author and title of the book

%----------------------------------------------------------------------------------------

\begin{document}

%----------------------------------------------------------------------------------------
%	TITLE PAGE
%----------------------------------------------------------------------------------------

\begingroup
\thispagestyle{empty} % Suppress headers and footers on the title page
\begin{tikzpicture}[remember picture,overlay]
\node[inner sep=0pt] (background) at (current page.center) {\includegraphics[width=\paperwidth]{background_unsa.pdf}};
\draw (current page.center) node [xshift=3cm]{\Huge\centering\bfseries\sffamily\parbox[c][][t]{\paperwidth - 8cm}
{\centering Universidad Nacional de San Agustín de Arequipa\\[20pt] % School
{\huge Escuela Profesional de Ciencia de la Computación}\\[20pt] % School
{\huge Computación Molecular Biológica}\\[1pt] % Course
{\huge (Código: 1005155)}\\[20pt] % Course
{\Large Semestre 2020A}}}; % Semester
\end{tikzpicture}
\vfill
\endgroup

\let\cleardoublepage\clearpage


%----------------------------------------------------------------------------------------
%	TABLE OF CONTENTS
%----------------------------------------------------------------------------------------

%\usechapterimagefalse % If you don't want to include a chapter image, use this to toggle images off - it can be enabled later with \usechapterimagetrue

\chapterimage{background_unsa.pdf} % Table of contents heading image

\pagestyle{empty} % Disable headers and footers for the following pages

\tableofcontents % Print the table of contents itself

\cleardoublepage % Forces the first chapter to start on an odd page so it's on the right side of the book

\pagestyle{fancy} % Enable headers and footers again

\chapterimage{background_unsa.pdf} % Chapter heading image

\import{chapter/}{texs/cv_icacit.tex}

\import{chapter/}{texs/silabo_dufa.tex}

\import{chapter/}{texs/silabo_icacit.tex}

\import{chapter/}{texs/prueba_entrada.tex}

\import{chapter/}{texs/primer_parcial.tex}

\import{chapter/}{texs/segundo_parcial.tex}

\import{chapter/}{texs/tercer_parcial.tex}

\import{chapter/}{texs/evaluacion_continua_1.tex}

\import{chapter/}{texs/evaluacion_continua_2.tex}

\import{chapter/}{texs/evaluacion_continua_3.tex}

\import{chapter/}{texs/asistencia.tex}

\import{chapter/}{texs/registro_notas.tex}

\import{chapter/}{texs/material_curso.tex}

\import{chapter/}{texs/desempeno.tex}

\import{chapter/}{texs/informe_final_curso.tex}

\end{document}
