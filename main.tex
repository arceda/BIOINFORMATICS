\documentclass[10pt]{beamer}
\usetheme[
%%% option passed to the outer theme
%    progressstyle=fixedCircCnt,   % fixedCircCnt, movingCircCnt (moving is deault)
  ]{Feather}
  
% If you want to change the colors of the various elements in the theme, edit and uncomment the following lines

% Change the bar colors:
%\setbeamercolor{Feather}{fg=red!20,bg=red}

% Change the color of the structural elements:
%\setbeamercolor{structure}{fg=red}

% Change the frame title text color:
%\setbeamercolor{frametitle}{fg=blue}

% Change the normal text color background:
%\setbeamercolor{normal text}{fg=black,bg=gray!10}

%-------------------------------------------------------
% INCLUDE PACKAGES
%-------------------------------------------------------

\usepackage[utf8]{inputenc}
\usepackage[english]{babel}
\usepackage[T1]{fontenc}
\usepackage{helvet}

%-------------------------------------------------------
% DEFFINING AND REDEFINING COMMANDS
%-------------------------------------------------------

% colored hyperlinks
\newcommand{\chref}[2]{
  \href{#1}{{\usebeamercolor[bg]{Feather}#2}}
}

%-------------------------------------------------------
% INFORMATION IN THE TITLE PAGE
%-------------------------------------------------------

\title[] % [] is optional - is placed on the bottom of the sidebar on every slide
{ % is placed on the title page
      \textbf{Introduction to Bioinformatics}
}

\subtitle[Introduction  to Bioinformatics]
{
      \textbf{Bioinformatics}
}

\author[MSc. Vicente Machaca Arceda]
{      MSc. Vicente Machaca Arceda \\
      {}
}

\institute[]
{
      Universidad Nacional de san Agustín de Arequipa
  
  %there must be an empty line above this line - otherwise some unwanted space is added between the university and the country (I do not know why;( )
}

\date{\today}

%-------------------------------------------------------
% THE BODY OF THE PRESENTATION
%-------------------------------------------------------

\begin{document}


\AtBeginSection[]
{
    \begin{frame}
        \frametitle{Table of Contents}
        \tableofcontents[currentsection]
    \end{frame}
}


%-------------------------------------------------------
% THE TITLEPAGE
%-------------------------------------------------------

{\1% % this is the name of the PDF file for the background


%-------------------------------------------------------
%-------------------------------------------------------
\begin{frame}{Introduction  to Bioinformatics}{}
\centering
\textbf{INTRODUCTION TO BIOINFORMATICS} \\
MSc. Vicente Machaca Arceda 
\end{frame}
%-------------------------------------------------------
%-------------------------------------------------------

%-------------------------------------------------------
%-------------------------------------------------------
\begin{frame}{Overview}
\tableofcontents
\end{frame}
%-------------------------------------------------------
%-------------------------------------------------------


\section{Introduction}

%%%%%%%%%%%%%%%%%%%%%%%%%%%%%%%%%%%%%%%%%%%%%%%%%%%%%%%%
\subsection{Objectives}
%%%%%%%%%%%%%%%%%%%%%%%%%%%%%%%%%%%%%%%%%%%%%%%%%%%%%%%%

%-------------------------------------------------------
%-------------------------------------------------------
\begin{frame}{Objectives}{}
\begin{itemize}
    \item<1-> Understand what is Bioinformatics, computer biology and computation molecular biology. 
    \item<2-> Understand about the biology of DNA.
    \item<3-> Understand how is the process of protein synthesis from DNA.
  \end{itemize}
\end{frame}
%-------------------------------------------------------
%-------------------------------------------------------

%%%%%%%%%%%%%%%%%%%%%%%%%%%%%%%%%%%%%%%%%%%%%%%%%%%%%%%%
\subsection{Motivation}
%%%%%%%%%%%%%%%%%%%%%%%%%%%%%%%%%%%%%%%%%%%%%%%%%%%%%%%%

%-------------------------------------------------------
%-------------------------------------------------------
\begin{frame}{Motivation}{What microorganism live in our armpits or in our mouths?}
\begin{figure}[]
 \centering
    \includegraphics[width=\textwidth,height=0.7\textheight,keepaspectratio]{img/mot1.jpg}
    \label{img:mot1}
    \caption{What microorganism live in our armpits or in our mouths?}
\end{figure}
\end{frame}
%-------------------------------------------------------
%-------------------------------------------------------

%-------------------------------------------------------
%-------------------------------------------------------
\begin{frame}{Motivation}{Is there a kindness gene?
}
\begin{figure}[]
 \centering
    \includegraphics[width=\textwidth,height=0.7\textheight,keepaspectratio]{img/mot2.jpg}
    \label{img:mot2}
    \caption{Is there a kindness gene?}
\end{figure}
\end{frame}
%-------------------------------------------------------
%-------------------------------------------------------

%-------------------------------------------------------
%-------------------------------------------------------
\begin{frame}{Motivation}{Why a person has cancer?}
\begin{figure}[]
 \centering
    \includegraphics[width=\textwidth,height=0.6\textheight,keepaspectratio]{img/mot3.jpg}
    \label{img:mot2}
    \caption{Why a person has cancer?}
\end{figure}
\end{frame}
%-------------------------------------------------------
%-------------------------------------------------------

%-------------------------------------------------------
%-------------------------------------------------------
\begin{frame}{Motivation}{Why some medicines no work in some persons?}
\begin{figure}[]
 \centering
    \includegraphics[width=\textwidth,height=0.6\textheight,keepaspectratio]{img/mot4.jpg}
    \label{img:mot2}
    \caption{Why some medicines no work in some persons?}
\end{figure}
\end{frame}
%-------------------------------------------------------
%-------------------------------------------------------

%-------------------------------------------------------
%-------------------------------------------------------
\begin{frame}{Motivation}{Treatment Development}
\begin{figure}[]
 \centering
    \includegraphics[width=\textwidth,height=0.7\textheight,keepaspectratio]{img/mot5.jpg}
    \label{img:mot2}
    \caption{Personalized Medicine: New Approach to Treatment of Disease}
\end{figure}
\end{frame}
%-------------------------------------------------------
%-------------------------------------------------------

%%%%%%%%%%%%%%%%%%%%%%%%%%%%%%%%%%%%%%%%%%%%%%%%%%%%%%%%
\subsection{What is Bioinformatics?}
%%%%%%%%%%%%%%%%%%%%%%%%%%%%%%%%%%%%%%%%%%%%%%%%%%%%%%%%

%-------------------------------------------------------
\begin{frame}{Introduction}{What is Bioinformatics?}
%-------------------------------------------------------

    According to Luscombe et al.: \textbf{Bioinformatics} involves the technology that uses computers for storage, retrieval, manipulation, and distribution of information related to biological macromolecules such as DNA, RNA, and proteins \cite{luscombe2001bioinformatics}.

\end{frame}

%-------------------------------------------------------
%-------------------------------------------------------
\begin{frame}{Introduction}{Bioinformatics vs Computational Biology}
\textbf{Bioinformatics} is limited to sequence, structural, and functional analysis of genes and genomes and their corresponding products and is often considered \textbf{Computational
molecular biology}. However, \textbf{Computational Biology} encompasses all biological areas that involve computation \cite{xiong2006essential}.
\end{frame}
%-------------------------------------------------------
%-------------------------------------------------------

%-------------------------------------------------------
%-------------------------------------------------------
\begin{frame}{Introduction}{Genomics}
\textbf{Genomics} is the study of whole genomes of organisms. Genomics uses a combination of recombinant DNA, DNA sequencing methods, and bioinformatics to sequence, assemble, and analyse the structure and function of genomes. It differs from classical \textbf{Genetics} in that it study genes and their heredity meanwhile Genomics study the whole genome  \cite{archibald2018genomics}.
\end{frame}
%-------------------------------------------------------
%-------------------------------------------------------


\section{The biology of cells}


%%%%%%%%%%%%%%%%%%%%%%%%%%%%%%%%%%%%%%%%%%%%%%%%%%%%%%%%
\subsection{Where is DNA?}
%%%%%%%%%%%%%%%%%%%%%%%%%%%%%%%%%%%%%%%%%%%%%%%%%%%%%%%%

%-------------------------------------------------------
%-------------------------------------------------------
\begin{frame}{The biology of cells}{Where is DNA?}
\begin{figure}[]
 \centering
    \includegraphics[width=\textwidth,height=0.6\textheight,keepaspectratio]{img/bio1.jpg}
    \label{img:mot2}
    \caption{Where DNA is located in prokaryote and eukaryote cells \cite{archibald2018genomics}.}
\end{figure}
\end{frame}
%-------------------------------------------------------
%-------------------------------------------------------


%-------------------------------------------------------
%-------------------------------------------------------
\begin{frame}{The biology of cells}{Where is DNA?}
\begin{figure}[]
 \centering
    \includegraphics[width=\textwidth,height=0.6\textheight,keepaspectratio]{img/bio2.jpg}
    \label{img:mot2}
    \caption{The 23 pairs of chromosomes in human cells \cite{archibald2018genomics}.}
\end{figure}
\end{frame}
%-------------------------------------------------------
%-------------------------------------------------------

%-------------------------------------------------------
%-------------------------------------------------------
\begin{frame}{The biology of cells}{Where is DNA?}
\begin{figure}[]
 \centering
    \includegraphics[width=\textwidth,height=0.6\textheight,keepaspectratio]{img/bio3.jpg}
    \label{img:mot2}
    \caption{Chromatin: Material composed of DNA and proteins that condense to form chromosomes \cite{archibald2018genomics}.}
\end{figure}
\end{frame}
%-------------------------------------------------------
%-------------------------------------------------------

%-------------------------------------------------------
%-------------------------------------------------------
\begin{frame}{The biology of cells}{Where is DNA?}
\begin{figure}[]
 \centering
    \includegraphics[width=\textwidth,height=0.6\textheight,keepaspectratio]{img/bio4.jpg}
    \label{img:mot2}
    \caption{Where DNA is located \cite{dna2020located}.}
\end{figure}
\end{frame}
%-------------------------------------------------------
%-------------------------------------------------------

%%%%%%%%%%%%%%%%%%%%%%%%%%%%%%%%%%%%%%%%%%%%%%%%%%%%%%%%
\subsection{DNA structure}
%%%%%%%%%%%%%%%%%%%%%%%%%%%%%%%%%%%%%%%%%%%%%%%%%%%%%%%%

%-------------------------------------------------------
%-------------------------------------------------------
\begin{frame}{The biology of cells}{DNA structure}
\begin{figure}[]
 \centering
    \includegraphics[width=\textwidth,height=0.7\textheight,keepaspectratio]{img/bio6.jpg}
    \label{img:mot2}
    \caption{Molecules in DNA. Adenine, Thymine, Guanine and Cytosine \cite{dna2020located}.}
\end{figure}
\end{frame}
%-------------------------------------------------------
%-------------------------------------------------------

%-------------------------------------------------------
%-------------------------------------------------------
\begin{frame}{The biology of cells}{DNA structure}
\begin{figure}[]
 \centering
    \includegraphics[width=\textwidth,height=0.7\textheight,keepaspectratio]{img/bio7.jpg}
    \label{img:mot2}
    \caption{DNA structure \cite{dnastructure2020}.}
\end{figure}
\end{frame}
%-------------------------------------------------------
%-------------------------------------------------------

%-------------------------------------------------------
%-------------------------------------------------------
\begin{frame}{The biology of cells}{DNA structure}
\begin{figure}[]
 \centering
    \includegraphics[width=\textwidth,height=0.6\textheight,keepaspectratio]{img/bio8.jpg}
    \label{img:mot2}
    \caption{Chromosome-DNA-gene \cite{dnacromosome2020}.}
\end{figure}
\end{frame}
%-------------------------------------------------------
%-------------------------------------------------------

%%%%%%%%%%%%%%%%%%%%%%%%%%%%%%%%%%%%%%%%%%%%%%%%%%%%%%%%
\subsection{Transcription and Translation}
%%%%%%%%%%%%%%%%%%%%%%%%%%%%%%%%%%%%%%%%%%%%%%%%%%%%%%%%

%-------------------------------------------------------
%-------------------------------------------------------
\begin{frame}{The biology of cells}{Transcription and Translation}
\begin{figure}[]
 \centering
    \includegraphics[width=\textwidth,height=0.6\textheight,keepaspectratio]{img/bio9.jpg}
    \label{img:mot2}
    \caption{Transcription and Translation \cite{dnacromosome2020}.}
\end{figure}
\end{frame}
%-------------------------------------------------------
%-------------------------------------------------------

%%%%%%%%%%%%%%%%%%%%%%%%%%%%%%%%%%%%%%%%%%%%%%%%%%%%%%%%
\subsection{From DNA to Protein}
%%%%%%%%%%%%%%%%%%%%%%%%%%%%%%%%%%%%%%%%%%%%%%%%%%%%%%%%

%-------------------------------------------------------
%-------------------------------------------------------
\begin{frame}{The biology of cells}{From DNA to Protein}
\href{https://www.youtube.com/watch?v=gG7uCskUOrA}{Click here to see the video} \\
\begin{figure}[]
 \centering
    \includegraphics[width=\textwidth,height=0.6\textheight,keepaspectratio]{img/bio10.jpg}
    \label{img:mot2}
    \caption{Video from DNA to protein.}
\end{figure}
\end{frame}
%-------------------------------------------------------
%-------------------------------------------------------

%-------------------------------------------------------
%-------------------------------------------------------
%\begin{frame}{Bioinformatics}{Homework}
%    Register to the following courses and bring yours certificated of accomplish: \\
%    \begin{itemize} 
%        \item \href{https://edu.t-bio.info/course/introduction-bioinformatics/}{Introduction to Bioinformatics (6 hours) } \\
%        \item \href{https://edu.t-bio.info/course/introduction-to-genomics/}{Introduction to Genomics (4 hours)}
%    \end{itemize}
%\end{frame}
%-------------------------------------------------------
%-------------------------------------------------------

%-------------------------------------------------------
%-------------------------------------------------------
\begin{frame}[allowframebreaks]
        \frametitle{References}
        %\bibliographystyle{amsalpha}
        \bibliographystyle{IEEEtran}
        \bibliography{bibliography.bib}
\end{frame}
%-------------------------------------------------------
%-------------------------------------------------------

%-------------------------------------------------------
%-------------------------------------------------------
{\1
\begin{frame}[plain,noframenumbering]
  \finalpage{Thank you}
\end{frame}}
%-------------------------------------------------------
%-------------------------------------------------------

%%%%%%%%%%%%%%%%%%%%%%%%%%%%%%%%%%%%%%%%%%%%%%%%%%%%%%%%%%%%%%%%%%%%%%%%%%%%%%%%%%%%%%%%%%%
%%%%%%%%%%%%%%%%%%%%%%%%%%%%%%%%%%%%%%%%%%%%%%%%%%%%%%%%%%%%%%%%%%%%%%%%%%%%%%%%%%%%%%%%%%%
%%%%%%%%%%%%%%%%%%%%%%%%%%%%%%%%%%%%%%%%%%%%%%%%%%%%%%%%%%%%%%%%%%%%%%%%%%%%%%%%%%%%%%%%%%%
%%%%%%%%%%%%%%%%%%%%%%%%%%%%%%%%%%%%%%%%%%%%%%%%%%%%%%%%%%%%%%%%%%%%%%%%%%%%%%%%%%%%%%%%%%%
\iffalse

%-------------------------------------------------------
\subsection{Local and Global installation}
\begin{frame}{Installation}{Local and Global installation}
%-------------------------------------------------------
  The theme can be installed for \textbf{local} or \textbf{global} use.
  \pause
  \begin{block}{Local Installation}
  \begin{itemize}    
    \item Local installation is the simplest way of installing the theme. 
    \item You need to placing the 4 source files in the same folder as your presentation. When you download the theme, the 4 theme files are located in the {\tt local} folder.
  \end{itemize}
  \end{block}

  \begin{block}{Global Installation}
  \begin{itemize}
     \item If you wish to make the theme globally available, you must put the files in your local latex directory tree. The location of the root of the local directory tree depends on your operating system and the latex distribution. 
     \item Detailed steps on how to proceed installation under various operating systems can be found at Beamer documentation.
  \end{itemize}
  \end{block}
\end{frame}
     

%-------------------------------------------------------
\subsection{Required Packages}
\begin{frame}{Installation}{Required Packages}
%-------------------------------------------------------

  For using the Feather Theme you will need the Bemaer class installed and the following 2 packages
  \begin{itemize}
    \item TikZ\footnote{TikZ is a package for creating beautiful graphics. Have a look at these \chref{http://www.texample.net/tikz/examples/}{online examples} or the \chref{http://tug.ctan.org/tex-archive/graphics/pgf/base/doc/generic/pgf/pgfmanual.pdf}{pgf user manual}.}
    \item calc
  \end{itemize}
  Due to the fact that the packages are very common they should be included in your latex distribution in the first place.
\end{frame}

%-------------------------------------------------------
\section{User Interface}
\subsection{Loading the Theme and Theme Options}
\begin{frame}{User Interface}{Loading the Theme and Theme Options}
%-------------------------------------------------------

  \begin{block}{The Presentation Theme}
    The Feather Theme can be loaded in a familiar way. In the reamble of your {\tt tex} file you must type\\ \vspace{5pt} 
    {\tt \textbackslash usetheme[<options>]\{Feather\}}\\ \vspace{5pt} 
    The presentation theme loads the inner, outer and color Feather theme files and passes the {\tt <options>} on to these files.
  \end{block}
  \begin{block}{The Inner and Outher Themes}
    If you wish you can load only the inner, or the outher theme directly by\\ \vspace{5pt} 
    {\tt \textbackslash useinnertheme\{Feather\}} (and it has no options)\\ \vspace{5pt} 
    {\tt \textbackslash useoutertheme[<options>]\{Feather\}} (it has one option)\\
    \hspace{20pt}{\tt progressstyle=\{fixedCircCnt or movingCircCnt\}} \\
    \begin{itemize}
    \item which set how the progress is illustrated;
    \item the value {\tt movingCircCnt} is the default.
    \end{itemize}
  \end{block}
\end{frame}

\begin{frame}{User Interface}{Loading the Theme and Theme Options}

  \begin{block}{The Color Theme}
    Also you can load only the color theme by writing in the preamble of the {\tt tex} file 
    
    \vspace{5pt} 
    
    \begin{itemize}
    \item {\tt \textbackslash usecolortheme\{Feather\}}
    \end{itemize}
    
    \vspace{5pt}
    
    ...or to change the colors of the various elements in the theme
    
    \vspace{5pt} 
    \begin{itemize}
    \item Change the bar colors: \\    
    {\tt \textbackslash setbeamercolor \{Feather\}\{fg=<color>, bg=<color>\}}
    
    \vspace{2pt} 
    
    \item Change the color of the structural elements: \\    
    {\tt \textbackslash setbeamercolor\{structure\}\{fg=<color>\}}
    
    \vspace{2pt} 
    
    \item Change the frame title text color:\\
    {\tt \textbackslash setbeamercolor\{frametitle\}\{fg=<color>\}}
    
    \vspace{2pt} 
    
    \item Change the normal text color background:    
    {\tt \textbackslash setbeamercolor\{normal text\}\{fg=<color>, bg=<color>\}}
    \end{itemize}
  \end{block}
\end{frame}


%-------------------------------------------------------
\subsection{Feather image}
\begin{frame}{User Interface}{The Feather Background Image}
%-------------------------------------------------------

\begin{block}{The Feather Background Image}
    \begin{itemize}
    \item In Feather theme, the title page frame and the last frame have the Feather image as the background image. 
    \item The Feather background image can be produced to any frame by wrating on the begining at the choosen frame the following
    \end{itemize} 
    
    \vspace{5pt} 
    
  {\tt \{\textbackslash 1bg\\
    \textbackslash begin\{frame\}[<options>]\{Frame Title\}\{Frame Subtitle\}\\
    \ldots\\
    \textbackslash end\{frame\}\}}
\end{block}
\end{frame}





\fi

\end{document}