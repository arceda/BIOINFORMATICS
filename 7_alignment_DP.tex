\documentclass[10pt]{beamer}
\usepackage[english]{babel}
\usepackage[utf8]{inputenc}
\usepackage[T1]{fontenc}
\usepackage{helvet}

%-------------------------------------------------------
% INFORMATION IN THE TITLE PAGE
%-------------------------------------------------------

\newcommand{\cstitle}{\textbf{Bioinformatics}}
\subtitle[]{Sequence alignment - Dinamic programming}
\newcommand{\cscourseCode}{1005155}
\newcommand{\csauthor}{MSc. Vicente Machaca Arceda}
\institute[UNSA]{Universidad Nacional de San Agustín de Arequipa}
\newcommand{\csemail}{vmachacaa@unsa.edu.pe}
\newcommand{\instituteabr}{UNSA}
\newcommand{\nameUp}{ICC Fase 1}
\date{\today}
\title[\cscourseCode]{\cstitle}
\author{\csauthor}
%%%%%%%%%%%%%%%%%

%-------------------------------------------------------
% CHOOSE THE THEME
%-------------------------------------------------------
\def\mycmd{0} % CS THEME
%\def\mycmd{1} % MYTHEME
%-------------------------------------------------------


\if\mycmd1
\usetheme[]{Feather}
\newcommand{\chref}[2]{	\href{#1}{{\usebeamercolor[bg]{Feather}#2}} }
\else
\usepackage{csformat}
\fi

\newcommand{\1}{
	\setbeamertemplate{background}{
		\includegraphics[width=\paperwidth,height=\paperheight]{img/1}
		\tikz[overlay] \fill[fill opacity=0.75,fill=white] (0,0) rectangle (-\paperwidth,\paperheight);
	}
}



%-------------------------------------------------------
% THE BODY OF THE PRESENTATION
%-------------------------------------------------------

\begin{document}
	
	
	\AtBeginSection[]
	{
		\begin{frame}
			\frametitle{Table of Contents}
			\tableofcontents[currentsection]
		\end{frame}
	}
	
	
	%-------------------------------------------------------
	% THE TITLEPAGE
	%-------------------------------------------------------
	
	\if\mycmd1 % MY THEME
	\1{
		\begin{frame}[plain,noframenumbering] 
			\titlepage 
		\end{frame}}
		\else % CS THEME
		\maketitle
		\fi

%-------------------------------------------------------
%-------------------------------------------------------
\begin{frame}{Table of Contents}
\tableofcontents
\end{frame}
%-------------------------------------------------------
%-------------------------------------------------------

%%%%%%%%%%%%%%%%%%%%%%%%%%%%%%%%%%%%%%%%%%%%%%%%%%%%%%%%%%%%%%%%%%%%%%%%%%%%%%%%%%%%%%%%%%%%%%%%%%%%%%%%%%%%%%%%
%%%%%%%%%%%%%%%%%%%%%%%%%%%%%%%%%%%%%%%%%%%%%%%%%%%%%%%%%%%%%%%%%%%%%%%%%%%%%%%%%%%%%%%%%%%%%%%%%%%%%%%%%%%%%%%%
\section{Introduction}
%%%%%%%%%%%%%%%%%%%%%%%%%%%%%%%%%%%%%%%%%%%%%%%%%%%%%%%%%%%%%%%%%%%%%%%%%%%%%%%%%%%%%%%%%%%%%%%%%%%%%%%%%%%%%%%%
%%%%%%%%%%%%%%%%%%%%%%%%%%%%%%%%%%%%%%%%%%%%%%%%%%%%%%%%%%%%%%%%%%%%%%%%%%%%%%%%%%%%%%%%%%%%%%%%%%%%%%%%%%%%%%%%

%%%%%%%%%%%%%%%%%%%%%%%%%%%%%%%%%%%%%%%%%%%%%%%%%%%%%%%%%%%%%%%%%%%%%%%%%%%%%%%%%%%%%%%%%%%%%%%%%%%%%%%%%%%%%%%%
%%%%%%%%%%%%%%%%%%%%%%%%%%%%%%%%%%%%%%%%%%%%%%%%%%%%%%%%%%%%%%%%%%%%%%%%%%%%%%%%%%%%%%%%%%%%%%%%%%%%%%%%%%%%%%%%
\subsection{Objectives}
%%%%%%%%%%%%%%%%%%%%%%%%%%%%%%%%%%%%%%%%%%%%%%%%%%%%%%%%%%%%%%%%%%%%%%%%%%%%%%%%%%%%%%%%%%%%%%%%%%%%%%%%%%%%%%%%
%%%%%%%%%%%%%%%%%%%%%%%%%%%%%%%%%%%%%%%%%%%%%%%%%%%%%%%%%%%%%%%%%%%%%%%%%%%%%%%%%%%%%%%%%%%%%%%%%%%%%%%%%%%%%%%%

%-------------------------------------------------------
%-------------------------------------------------------
\begin{frame}{Introduction}{Objectives}
\begin{itemize}
    \item<1-> Understand the importance of sequence alignment in Bioinformatics. 
    \item<2-> Understand and implement the Needleman–Wunsch algorithm.
  \end{itemize}
\end{frame}
%-------------------------------------------------------
%-------------------------------------------------------



%%%%%%%%%%%%%%%%%%%%%%%%%%%%%%%%%%%%%%%%%%%%%%%%%%%%%%%%%%%%%%%%%%%%%%%%%%%%%%%%%%%%%%%%%%%%%%%%%%%%%%%%%%%%%%%%
%%%%%%%%%%%%%%%%%%%%%%%%%%%%%%%%%%%%%%%%%%%%%%%%%%%%%%%%%%%%%%%%%%%%%%%%%%%%%%%%%%%%%%%%%%%%%%%%%%%%%%%%%%%%%%%%
\section{Sequence alignment}
%%%%%%%%%%%%%%%%%%%%%%%%%%%%%%%%%%%%%%%%%%%%%%%%%%%%%%%%%%%%%%%%%%%%%%%%%%%%%%%%%%%%%%%%%%%%%%%%%%%%%%%%%%%%%%%%
%%%%%%%%%%%%%%%%%%%%%%%%%%%%%%%%%%%%%%%%%%%%%%%%%%%%%%%%%%%%%%%%%%%%%%%%%%%%%%%%%%%%%%%%%%%%%%%%%%%%%%%%%%%%%%%%

%%%%%%%%%%%%%%%%%%%%%%%%%%%%%%%%%%%%%%%%%%%%%%%%%%%%%%%%%%%%%%%%%%%%%%%%%%%%%%%%%%%%%%%%%%%%%%%%%%%%%%%%%%%%%%%%
%%%%%%%%%%%%%%%%%%%%%%%%%%%%%%%%%%%%%%%%%%%%%%%%%%%%%%%%%%%%%%%%%%%%%%%%%%%%%%%%%%%%%%%%%%%%%%%%%%%%%%%%%%%%%%%%
\subsection{Definition}
%%%%%%%%%%%%%%%%%%%%%%%%%%%%%%%%%%%%%%%%%%%%%%%%%%%%%%%%%%%%%%%%%%%%%%%%%%%%%%%%%%%%%%%%%%%%%%%%%%%%%%%%%%%%%%%%
%%%%%%%%%%%%%%%%%%%%%%%%%%%%%%%%%%%%%%%%%%%%%%%%%%%%%%%%%%%%%%%%%%%%%%%%%%%%%%%%%%%%%%%%%%%%%%%%%%%%%%%%%%%%%%%%

%-------------------------------------------------------
%-------------------------------------------------------
\begin{frame}{Sequence alignment}{Definition}

\begin{block}{}
\centering
\textbf{How can we determine the similarity between two sequences?}
\end{block}

\end{frame}
%-------------------------------------------------------
%-------------------------------------------------------

%-------------------------------------------------------
%-------------------------------------------------------
\begin{frame}{Sequence alignment}{Definition}
	
	\begin{block}{Sequence Alignment in Biology}
		\centering
		The purpose of a sequence alignment is to line
		up all residues in the inputted sequence(s) for
		maximal level of similarity, in the sense of their
		functional or evolutionary relationship.
	\end{block}
	
\end{frame}
%-------------------------------------------------------
%-------------------------------------------------------


%%%%%%%%%%%%%%%%%%%%%%%%%%%%%%%%%%%%%%%%%%%%%%%%%%%%%%%%%%%%%%%%%%%%%%%%%%%%%%%%%%%%%%%%%%%%%%%%%%%%%%%%%%%%%%%%
%%%%%%%%%%%%%%%%%%%%%%%%%%%%%%%%%%%%%%%%%%%%%%%%%%%%%%%%%%%%%%%%%%%%%%%%%%%%%%%%%%%%%%%%%%%%%%%%%%%%%%%%%%%%%%%%
\subsection{Example}
%%%%%%%%%%%%%%%%%%%%%%%%%%%%%%%%%%%%%%%%%%%%%%%%%%%%%%%%%%%%%%%%%%%%%%%%%%%%%%%%%%%%%%%%%%%%%%%%%%%%%%%%%%%%%%%%
%%%%%%%%%%%%%%%%%%%%%%%%%%%%%%%%%%%%%%%%%%%%%%%%%%%%%%%%%%%%%%%%%%%%%%%%%%%%%%%%%%%%%%%%%%%%%%%%%%%%%%%%%%%%%%%%


%-------------------------------------------------------
%-------------------------------------------------------
\begin{frame}{Sequence alignment}{Pairwise sequence alignment}	
	Visit \href{https://www.ebi.ac.uk/Tools/psa/emboss_needle/}{\color{blue}{\underline{EMBOSS}}}, use the sample  sequences and evaluate with BLOSUM62 matrix (no more that 62\% of similarity).
	
	\begin{figure}[]
		\centering
		\includegraphics[width=\textwidth,height=0.6\textheight,keepaspectratio]{img/alignment/dp_1}
		\label{img:uniprot}
		\caption{Tool for pairwise sequence alignment.}
	\end{figure}
\end{frame}
%-------------------------------------------------------
%-------------------------------------------------------

%-------------------------------------------------------
%-------------------------------------------------------
\begin{frame}{Sequence alignment}{Pairwise sequence alignment}		
	\begin{figure}[]
		\centering
		\includegraphics[width=\textwidth,height=0.6\textheight,keepaspectratio]{img/alignment/dp_2}
		\label{img:uniprot}
		\caption{Alignment: ``|'' stands for equality, ``:'' for similarity and ``.'' for non-similarity.}
	\end{figure}
\end{frame}
%-------------------------------------------------------
%-------------------------------------------------------


%%%%%%%%%%%%%%%%%%%%%%%%%%%%%%%%%%%%%%%%%%%%%%%%%%%%%%%%%%%%%%%%%%%%%%%%%%%%%%%%%%%%%%%%%%%%%%%%%%%%%%%%%%%%%%%%
%%%%%%%%%%%%%%%%%%%%%%%%%%%%%%%%%%%%%%%%%%%%%%%%%%%%%%%%%%%%%%%%%%%%%%%%%%%%%%%%%%%%%%%%%%%%%%%%%%%%%%%%%%%%%%%%
\subsection{Maths}
%%%%%%%%%%%%%%%%%%%%%%%%%%%%%%%%%%%%%%%%%%%%%%%%%%%%%%%%%%%%%%%%%%%%%%%%%%%%%%%%%%%%%%%%%%%%%%%%%%%%%%%%%%%%%%%%
%%%%%%%%%%%%%%%%%%%%%%%%%%%%%%%%%%%%%%%%%%%%%%%%%%%%%%%%%%%%%%%%%%%%%%%%%%%%%%%%%%%%%%%%%%%%%%%%%%%%%%%%%%%%%%%%

%-------------------------------------------------------
%-------------------------------------------------------
\begin{frame}{Pairwise Sequence Alignment}{in Maths}		
	\begin{block}{}
	Input data:
			\begin{itemize}
				\item Two sequences $S_1$ and $S_2$
			\end{itemize}
	\end{block}

	\begin{block}{}
		Parameters:
			\begin{itemize}
				\item A scoring function $f$ for \textbf{substitutions} and \textbf{gaps}.
			\end{itemize}
	\end{block}

 	\begin{block}{}
 		Output:
			\begin{itemize}
				\item The optimal alignment of $S_1$ and $S_2$, which has the maximal
				score.
			\end{itemize}
	\end{block}
\end{frame}
%-------------------------------------------------------
%-------------------------------------------------------


%-------------------------------------------------------
%-------------------------------------------------------
\begin{frame}{Pairwise Sequence Alignment}{in Maths}
	There are too many possible solution for sequence alignment.		
	\begin{figure}[]
	\centering
		\includegraphics[width=\textwidth,height=0.3\textheight,keepaspectratio]{img/alignment/dp_3}
		\label{img:uniprot}
		%\caption{Alignment: ``|'' stands for equality, ``:'' for similarity and ``.'' for non-similarity.}
	\end{figure}

	\[
		\binom{2n}{n} = \frac{ (2n)! }{(n!)^2}
	\]

\end{frame}
%-------------------------------------------------------
%-------------------------------------------------------

%-------------------------------------------------------
%-------------------------------------------------------
\begin{frame}{Pairwise Sequence Alignment}{in Maths}	
	if $n=300$
	\[
	\binom{2n}{n} = \frac{ (2n)! }{(n!)^2} = \frac{ 2*300 }{(300!)^2}  \approx 7*10^{88}
	\]
	
	The visible universe is estimated to contain $10^{78} \sim 10^{80}$ atoms.
	
\end{frame}
%-------------------------------------------------------
%-------------------------------------------------------


%-------------------------------------------------------
%-------------------------------------------------------
\begin{frame}{Pairwise Sequence Alignment}{Dynamic Programming}	
	\begin{block}{Dynamic Programming}
		Dynamic Programming solves problems by combining the solutions to sub‐problems.
		\begin{itemize}
			\item Break the problem into smaller sub‐problems.
			\item Solve these sub‐problems optimally recursively.
			\item Use these optimal solutions to construct an optimal solution for the original problem.
		\end{itemize}
	\end{block}
\end{frame}
%-------------------------------------------------------
%-------------------------------------------------------

%-------------------------------------------------------
%-------------------------------------------------------
\begin{frame}{Dynamic Programming}{Example}		
	\begin{figure}[]
		\centering
		\includegraphics[width=0.5\textwidth,keepaspectratio]{img/alignment/dp_5}
		\label{img:uniprot}
		\caption{How we could divide the problem into sub-problems.}
	\end{figure}	

\end{frame}
%-------------------------------------------------------
%-------------------------------------------------------


%-------------------------------------------------------
%-------------------------------------------------------
\begin{frame}{Dynamic Programming}{Example}		
	\begin{figure}[]
		\centering
		\includegraphics[width=\textwidth,keepaspectratio]{img/alignment/dp_4}
		\label{img:uniprot}
		%\caption{How we could divide the problem into sub-problems.}
	\end{figure}	
	
\end{frame}
%-------------------------------------------------------
%-------------------------------------------------------

%-------------------------------------------------------
%-------------------------------------------------------
\begin{frame}{Dynamic Programming}{Example}			
	\begin{figure}[]
		\centering
		\includegraphics[width=\textwidth,keepaspectratio]{img/alignment/dp_6}
		\label{img:uniprot}
		\caption{Example of $s(x_i, y_i)$.}
	\end{figure}	
	
\end{frame}
%-------------------------------------------------------
%-------------------------------------------------------

%-------------------------------------------------------
%-------------------------------------------------------
\begin{frame}{Dynamic Programming}{Example}	
	\begin{figure}[]
		\centering
		\includegraphics[width=\textwidth,keepaspectratio]{img/alignment/dp_7}
		\label{img:uniprot}
		%\caption{Example of $s(x_i, y_i)$.}
	\end{figure}	
	
\end{frame}
%-------------------------------------------------------
%-------------------------------------------------------

%-------------------------------------------------------
%-------------------------------------------------------
\begin{frame}{Dynamic Programming}{Example}	
	\begin{figure}[]
		\centering
		\includegraphics[width=1.05\textwidth,keepaspectratio]{img/alignment/dp_8}
		\label{img:uniprot}
		%\caption{Example of $s(x_i, y_i)$.}
	\end{figure}	
	
\end{frame}
%-------------------------------------------------------
%-------------------------------------------------------

%-------------------------------------------------------
%-------------------------------------------------------
\begin{frame}{Dynamic Programming}{Example}	
	\begin{figure}[]
		\centering
		\includegraphics[width=1.05\textwidth,keepaspectratio]{img/alignment/dp_9}
		\label{img:uniprot}
		%\caption{Example of $s(x_i, y_i)$.}
	\end{figure}	
	
\end{frame}
%-------------------------------------------------------
%-------------------------------------------------------

%-------------------------------------------------------
%-------------------------------------------------------
\begin{frame}{Dynamic Programming}{Example}	
	\begin{figure}[]
		\centering
		\includegraphics[width=1.05\textwidth,keepaspectratio]{img/alignment/dp_10}
		\label{img:uniprot}
		%\caption{Example of $s(x_i, y_i)$.}
	\end{figure}	
	
\end{frame}
%-------------------------------------------------------
%-------------------------------------------------------

%-------------------------------------------------------
%-------------------------------------------------------
\begin{frame}{Dynamic Programming}{Example}	
	\begin{figure}[]
		\centering
		\includegraphics[width=1.05\textwidth,keepaspectratio]{img/alignment/dp_11}
		\label{img:uniprot}
		%\caption{Example of $s(x_i, y_i)$.}
	\end{figure}	
	
\end{frame}
%-------------------------------------------------------
%-------------------------------------------------------


%-------------------------------------------------------
%-------------------------------------------------------
\begin{frame}{Dynamic Programming}{Example}	
	Trace back to the upper left. Each arrow introduces
	one symbol at the end of each aligned sequence.
	\begin{figure}[]
		\centering
		\includegraphics[width=1.05\textwidth,keepaspectratio]{img/alignment/dp_12}
		\label{img:uniprot}
		%\caption{Example of $s(x_i, y_i)$.}
	\end{figure}	
	
\end{frame}
%-------------------------------------------------------
%-------------------------------------------------------


%-------------------------------------------------------
%-------------------------------------------------------
\if\mycmd1 % MY THEME
\1{
	{\1
		\begin{frame}[plain,noframenumbering]
			%\finalpage{Thank you}
			\begin{figure}[]
				\centering
				\includegraphics[width=\textwidth,height=0.7\textheight,keepaspectratio]{img/question.png}
				%\label{img:mot2}
				%\caption{Image example in 2 gray levels.}
			\end{figure}
	\end{frame}}
	\else % CS THEME
	\begin{frame}{Questions?}
		\begin{figure}[]
			\centering
			\includegraphics[width=\textwidth,height=0.7\textheight,keepaspectratio]{img/question.png}
			%\label{img:mot2}
			%\caption{Image example in 2 gray levels.}
		\end{figure}
		
	\end{frame}
	\fi
	%-------------------------------------------------------
	%-------------------------------------------------------



\end{document}