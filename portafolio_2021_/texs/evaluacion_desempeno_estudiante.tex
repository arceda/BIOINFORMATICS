\chapter{Evaluación del Desempeño del Estudiante}
\newpage
\pagestyle{empty} % Disable headers and footers for the following pages

\newcolumntype{L}[1]{>{\raggedright\arraybackslash}m{#1}}
\newcolumntype{C}[1]{>{\centering\arraybackslash}m{#1}}
\newcolumntype{R}[1]{>{\raggedleft\arraybackslash}m{#1}}

\section{Mapeo del Resultado del Estudiante}
\begin{itemize}
\item \lbrack a\rbrack ~Conocimientos en Computación: Nivel 2.
\item \lbrack c\rbrack ~Diseño y Desarrollo de Soluciones: Nivel 2.
\item \lbrack d\rbrack ~Trabajo Individual y en Equipo: Nivel 1.
\item \lbrack e\rbrack ~Comunicación: Nivel 1.
\end{itemize}

\section{Planificación de las Evaluaciones}

%------------------ RE a

\begin{landscape}
\subsection{Resultado del Estudiante \lbrack a\rbrack:}
Conocimientos de Computación: La capacidad de aplicar conocimientos de matemáticas, ciencias, computación y una especialidad de computación apropiados para los resultados del estudiante y la disciplina del programa. \textbf{Nivel 2 (Aplica a Nivel Intermedio)}

\begin{table}[h]
\centering
\begin{tabular}{C{5cm}|C{2cm}|C{2cm}|C{5cm}|C{2cm}|C{2cm}|C{2cm}}
\hline
\textbf{Indicador de Desempeño} & 
\textbf{Curso} & 
\textbf{Métodos - Assessment} & 
\textbf{Fuentes de Assessment (curso, semana y actividad)} & 
\textbf{Ciclo de Assessment} & 
\textbf{Coordinador Assesment} & 
\textbf{Nivel de Logro Esperado}
\\ \hline
a1. Demuestra conocimiento de los Algoritmos dentro de la Rama de Computación Evolutiva &
Computación Bioinspirada &
Rúbrica (Directa) &
\makecell{Computación Bioinspirada \\ Semana 7 \\ Primer Evaluación Parcial} &
2019-A &
Edward Hinojosa C. &
60\% 
\\ \hline
a2. Demuestra conocimiento de los Algoritmos dentro de la Rama de Computación Social &
Computación Bioinspirada &
Rúbrica (Directa) &
\makecell{Computación Bioinspirada \\ Semana 12 \\ Segunda Evaluación Parcial} &
2019-A &
Edward Hinojosa C. &
60\%
\\ \hline
\end{tabular}
\caption{Indicadores de Desempeño del Resultado del Estudiante \lbrack a\rbrack}
\label{tab:nivel_a}
\end{table}

\newpage

\begin{table}[h]
\centering
\begin{tabular}{C{5cm}|C{4cm}|C{4cm}|C{4cm}|C{4cm}}
\hline
\textbf{Indicador de Desempeño} & 
\textbf{1. Insatisfactorio} & 
\textbf{2. Parcialmente Satisfactorio} & 
\textbf{3. Satisfactorio} & 
\textbf{4. Excelente} 
\\ \hline
a1. Demuestra conocimiento de los Algoritmos dentro de la Rama de Computación Evolutiva &
No demuestra conocimiento de los Algoritmos dentro de la Rama de Computación Evolutiva vistos en el curso &
Demuestra conocimiento de menos del 30\% de los Algoritmos dentro de la Rama de Computación Evolutiva vistos en el curso &
Demuestra conocimiento de más del 30\% y menos del 80\% de los Algoritmos dentro de la Rama de Computación Evolutiva vistos en el curso &
Demuestra conocimiento de más 80\% de los Algoritmos dentro de la Rama de Computación Evolutiva vistos en el curso
\\ \hline
a2. Demuestra conocimiento de los Algoritmos dentro de la Rama de Computación Social &
No demuestra conocimiento de los Algoritmos dentro de la Rama de Computación Social vistos en el curso &
Demuestra conocimiento de menos del 30\% de los Algoritmos dentro de la Rama de Computación Social vistos en el curso &
Demuestra conocimiento de más del 30\% y menos del 80\% de los Algoritmos dentro de la Rama de Social Evolutiva vistos en el curso &
Demuestra conocimiento de más 80\% de los Algoritmos dentro de la Rama de Computación Social vistos en el curso
\\ \hline
\end{tabular}
\caption{Rúbrica a Usarse para cada Nivel del Logro en el Resultado del Estudiante \lbrack a\rbrack}
\label{tab:nivel_rubrica_a} 
\end{table}

\newpage

\begin{table}[h]
\centering
\begin{tabular}{L{6cm}|C{1cm}|C{1cm}|C{2cm}}
\hline
\textbf{Nombre} & 
\textbf{a1} & 
\textbf{a2} & 
\textbf{Evidencia} 
\\ \hline
Amable Romero, Diego Javier &
3 &
4 &
\makecell{\href{https://drive.google.com/open?id=1mBq4Xgh_DU0JrsuhhULiFpr2DSANNkTR}{Link} \\
\href{https://drive.google.com/open?id=1pMWVNwmbCAlMKkkZ20HA2tJ4Cz9KOybp}{Link}}
\\ \hline
Bernal Chahuayo, Luis Antonio &
3 &
3 &
\makecell{\href{https://drive.google.com/open?id=1kMVZjTaRL7Ljm2k2zBwHt4XDsJgJvkre}{Link} \\
\href{https://drive.google.com/open?id=177e8FGFaXUCIT2CRNO_i-DiU06sjkQH7}{Link}}
\\ \hline
Caceres Zegarra, Luis Gustavo &
3 &
3 &
\makecell{\href{https://drive.google.com/open?id=19LATNj6UQholxNT0LMHDvWSt7VgTWK09}{Link} \\
\href{https://drive.google.com/open?id=1eVcAY1rwdjbxuOwQviGR0tYqKSX-VrGD}{Link}}
\\ \hline
Espinel Quispe, Ingrid Sally &
2 &
1 &
\makecell{\href{https://drive.google.com/open?id=1XeLwif6D4CGuo_WrtgsbFCWoXFFjQSdh}{Link} \\
\href{https://drive.google.com/open?id=1tdDHdv3b46FVcuoXfkDbHvityP70oWh7}{Link}}
\\ \hline
Gordillo Viña, Karen &
3 &
3 &
\makecell{\href{https://drive.google.com/open?id=1-ObMxSUyya48modaI_jW5Jl_IySXUgx6}{Link} \\
\href{https://drive.google.com/open?id=1ecqq7fwLKZL_8FLpzCGptllY4I3oc1so}{Link}}
\\ \hline
Gutierrez Salazar, Enrique Alonzo &
3 &
4 &
\makecell{\href{https://drive.google.com/open?id=19SVz88sPCJ6AEGqKrwpoSGqIT6R8zkRL}{Link} \\
\href{https://drive.google.com/open?id=1qV4qk05ky-xGPGJethajAdj1_zb7YUtA}{Link}}
\\ \hline
Hancco Tancayllo, Hermith &
3 &
3 &
\makecell{\href{https://drive.google.com/open?id=1KMfoSp6P2eVN0WDdC_cChbnBxOqMm0I2}{Link} \\
\href{https://drive.google.com/open?id=1_YK49t-_ge6nm_LrBNI-0m6FXANRIxAt}{Link}}
\\ \hline
Huaman Canqui Jair, Francesco &
3 &
3 &
\makecell{\href{https://drive.google.com/open?id=13YAyoJTBXGwDtNPp9B5pneLk0WLPuTbJ}{Link} \\
\href{https://drive.google.com/open?id=1-OCEerOcv43sP_aU34rc4MLzQVfceVqZ}{Link}}
\\ \hline
Lacuaña Apaza, Margarita &
3 &
2 &
\makecell{\href{https://drive.google.com/open?id=1fHsXOvWvZyYJangde-xYXzjFwW8gnl5i}{Link} \\
\href{https://drive.google.com/open?id=1tQUwWS700bsLdf5w4zHMgHvI45HGMaNF}{Link}}
\\ \hline
Larraondo Lancho, Alejandro Jesús &
3 &
4 &
\makecell{\href{https://drive.google.com/open?id=1NyfPBkKibW1wTJnevLliKPPJYWpz8LNP}{Link} \\
\href{https://drive.google.com/open?id=1gHoNNeh1olKAMd4VEmBgero5vv3do5KI}{Link}}
\\ \hline
Mamani Chirinos, Luis &
1 &
1 &
\makecell{\href{https://drive.google.com/open?id=14YTJSk6yk4YdEjQHJPz1tJemxmfybIfT}{Link} \\
\href{https://drive.google.com/open?id=1HS19DQegS3izRw6_yOG-zwSM13TGqBIq}{Link}}
\\ \hline
Mendoza Villarroel, Alexis &
3 &
4 &
\makecell{\href{https://drive.google.com/open?id=1XTukrpi4u0-JAlxkfD_MFALN-Pfeax_N}{Link} \\
\href{https://drive.google.com/open?id=1V_T9FFp58mzFjYLbZTlKu7S_lvrgFJe1}{Link}}
\\ \hline
Quincho Mamani, Lehi &
3 &
1 &
\makecell{\href{https://drive.google.com/open?id=1IxMQEQ9kJg29mnHiJG0tzxSNDgq3b5Fr}{Link} \\
\href{https://drive.google.com/open?id=1c6taLWJSFkWHZmj0zQziyrWUinz3-A8m}{Link}}
\\ \hline
Quispe Quicano, Julio Cesar &
3 &
3 &
\makecell{\href{https://drive.google.com/open?id=1D0nX5t0eV9xsTW2-eMyXMsny84bEkgl9}{Link} \\
\href{https://drive.google.com/open?id=1BapaIzxoX-3Nd3rEekwmFkYuOK9pmUPo}{Link}}
\\ \hline
Turpo Apaza, Crhistian Andrew &
3 &
4 &
\makecell{\href{https://drive.google.com/open?id=1Ugp1_RgiQJdmYEi9PXYGRhFxJnh3zfUk}{Link} \\
\href{https://drive.google.com/open?id=1Fg3SU_YOXJaY4Mqb28qrCIKOdIdZHix4}{Link}}
\\ \hline
Uñapilco Chambi, Katherine &
3 &
3 &
\makecell{\href{https://drive.google.com/open?id=1oCXV7S2eM_tnFVeRYDV-isyQIEoBM2Le}{Link} \\
\href{https://drive.google.com/open?id=16Cp0sT3WmDwVarLYgWCnLJX9JISUqpoc}{Link}}
\\ \hline
\end{tabular}
\caption{Nivel del Logro para Cada Estudiante en el Resultado del Estudiante \lbrack a\rbrack}
\label{tab:nivel_estudiante_a} 
\end{table}

\newpage

\begin{table}[h]
\centering
\begin{tabular}{C{2cm}|C{2.5cm}|C{2.5cm}|C{2.5cm}|C{2.5cm}|C{4cm}}
\hline
\textbf{Indicador} & 
\textbf{Nivel 1 Insatisfactorio} & 
\textbf{Nivel 2 Parcialmente Satisfactorio} & 
\textbf{Nivel 3 Satisfactorio} & 
\textbf{Nivel 4 Excelente} &
\textbf{Nivel del Logro 2019A} 
\\ \hline
a1 &
\makecell{1 Estudiante(s) \\ (6.25\%)} &
\makecell{1 Estudiante(s) \\ (6.25\%)} &
\makecell{14 Estudiante(s) \\ (87.75\%)} &
\makecell{0 Estudiante(s) \\ (0.00\%)} &
\makecell{14 Estudiante(s) \\ (87.75\%) \\ > Objetivo (60\%)}
\\ \hline
a2 &
\makecell{3 Estudiante(s) \\ (18.75\%)} &
\makecell{1 Estudiante(s) \\ (6.25\%)} &
\makecell{7 Estudiante(s) \\ (43.75\%)} &
\makecell{5 Estudiante(s) \\ (31.25\%)} &
\makecell{12 Estudiante(s) \\ (75.00\%) \\ > Objetivo (60\%)}
\\ \hline
\end{tabular}
\caption{Resumen de los Niveles del Logro en el Resultado del Estudiante \lbrack a\rbrack}
\label{tab:nivel_resumen_a}
\end{table}

\end{landscape}

\begin{figure}
\centering
\begin{tikzpicture}[scale=1.00]
\pie [rotate = 180, 
	color={red,orange,yellow}]
    {6.25/Insatisfactorio,
     6.25/Parcialmente Satisfactorio,
     87.5/Satisfactorio}
\end{tikzpicture}
\caption{Resumen de los Niveles del Logro en el Resultado del Estudiante \lbrack a1\rbrack}
\end{figure}

\begin{figure}
\centering
\begin{tikzpicture}[scale=1.00]
\pie [rotate = 180, 
	color={red,orange,yellow,green}]
    {18.75/Insatisfactorio,
     6.25/Parcialmente Satisfactorio,
     43.75/Satisfactorio,
     31.25/Excelente}
\end{tikzpicture}
\caption{Resumen de los Niveles del Logro en el Resultado del Estudiante \lbrack a2\rbrack}
\end{figure}

%------------------------------------------------ Fin de RE a

%------------------ RE c

\begin{landscape}
\subsection{Resultado del Estudiante \lbrack c\rbrack:}
Diseño y Desarrollo de Soluciones: La capacidad de diseñar, implementar y evaluar soluciones a problemas complejos de computación y diseña y evalúa sistemas, componentes o procesos que satisfacen las necesidades específicas. \textbf{Nivel 2 (Aplica a Nivel Intermedio)}

\begin{table}[h]
\centering
\begin{tabular}{C{5cm}|C{2cm}|C{2cm}|C{5cm}|C{2cm}|C{2cm}|C{2cm}}
\hline
\textbf{Indicador de Desempeño} & 
\textbf{Curso} & 
\textbf{Métodos - Assessment} & 
\textbf{Fuentes de Assessment (curso, semana y actividad)} & 
\textbf{Ciclo de Assessment} & 
\textbf{Coordinador Assesment} & 
\textbf{Nivel de Logro Esperado}
\\ \hline
c1. Resuelve problemas implementado Algoritmos dentro de la Rama de Computación Evolutiva &
Computación Bioinspirada &
Rúbrica (Directa) &
\makecell{Computación Bioinspirada \\ Semana de la 2 a la 7 \\ Lab. 1, 2, 3, 4, 5, 6 , 7, 8} &
2019-A &
Edward Hinojosa C. &
60\% 
\\ \hline
c2. Resuelve problemas implementado Algoritmos dentro de la Rama de Computación Social &
Computación Bioinspirada &
Rúbrica (Directa) &
\makecell{Computación Bioinspirada \\ Semana de la 8 a la 12 \\ Lab. 9, 10, 11, 12, 13} &
2019-A &
Edward Hinojosa C. &
60\%
\\ \hline
\end{tabular}
\caption{Indicadores de Desempeño del Resultado del Estudiante \lbrack a\rbrack}
\label{tab:nivel_c}
\end{table}

\newpage

\begin{table}[h]
\centering
\begin{tabular}{C{5cm}|C{4cm}|C{4cm}|C{4cm}|C{4cm}}
\hline
\textbf{Indicador de Desempeño} & 
\textbf{1. Insatisfactorio} & 
\textbf{2. Parcialmente Satisfactorio} & 
\textbf{3. Satisfactorio} & 
\textbf{4. Excelente} 
\\ \hline
c1. Resuelve problemas implementado Algoritmos dentro de la Rama de Computación Evolutiva &
No resuelve problemas implementando los Algoritmos dentro de la Rama de Computación Evolutiva vistos en el curso &
Resuelve problemas implementando menos del 30\% de los Algoritmos dentro de la Rama de Computación Evolutiva vistos en el curso &
Resuelve problemas implementando más del 30\% y menos del 80\% de los Algoritmos dentro de la Rama de Computación Evolutiva vistos en el curso &
Resuelve problemas implementando más del 80\% de los Algoritmos dentro de la Rama de Computación Evolutiva vistos en el curso
\\ \hline
c2. Resuelve problemas implementado Algoritmos dentro de la Rama de Computación Social &
No resuelve problemas implementando los Algoritmos dentro de la Rama de Computación Social en el curso &
Resuelve problemas implementando menos del 30\% de los Algoritmos dentro de la Rama de Computación Social vistos en el curso &
Resuelve problemas implementando más del 30\% y menos del 80\% de los Algoritmos dentro de la Rama de Computación Social vistos en el curso &
Resuelve problemas implementando más del 80\% de los Algoritmos dentro de la Rama de Computación Social vistos en el curso
\\ \hline
\end{tabular}
\caption{Rúbrica a Usarse para cada Nivel del Logro en el Resultado del Estudiante \lbrack c\rbrack}
\label{tab:nivel_rubrica_c} 
\end{table}

\newpage

\begin{table}[h]
\centering
\begin{tabular}{L{6cm}|C{1cm}|C{1cm}|C{2cm}}
\hline
\textbf{Nombre} & 
\textbf{c1} & 
\textbf{c2} & 
\textbf{Evidencia} 
\\ \hline
Amable Romero, Diego Javier &
4 &
4 &
\href{https://drive.google.com/open?id=1a_VwNehS8UytMQLUPx3MKqE_XIXwuXfB}{Link}
\\ \hline
Bernal Chahuayo, Luis Antonio &
3 &
3 &
\href{https://drive.google.com/open?id=1a_VwNehS8UytMQLUPx3MKqE_XIXwuXfB}{Link}
\\ \hline
Caceres Zegarra, Luis Gustavo &
2 &
3 &
\href{https://drive.google.com/open?id=1a_VwNehS8UytMQLUPx3MKqE_XIXwuXfB}{Link}
\\ \hline
Espinel Quispe, Ingrid Sally &
3 &
2 &
\href{https://drive.google.com/open?id=1a_VwNehS8UytMQLUPx3MKqE_XIXwuXfB}{Link}
\\ \hline
Gordillo Viña, Karen &
4 &
4 &
\href{https://drive.google.com/open?id=1a_VwNehS8UytMQLUPx3MKqE_XIXwuXfB}{Link}
\\ \hline
Gutierrez Salazar, Enrique Alonzo &
4 &
4 &
\href{https://drive.google.com/open?id=1a_VwNehS8UytMQLUPx3MKqE_XIXwuXfB}{Link}
\\ \hline
Hancco Tancayllo, Hermith &
4 &
4 &
\href{https://drive.google.com/open?id=1a_VwNehS8UytMQLUPx3MKqE_XIXwuXfB}{Link}
\\ \hline
Huaman Canqui Jair, Francesco &
4 &
4 &
\href{https://drive.google.com/open?id=1a_VwNehS8UytMQLUPx3MKqE_XIXwuXfB}{Link}
\\ \hline
Lacuaña Apaza, Margarita &
4 &
4 &
\href{https://drive.google.com/open?id=1a_VwNehS8UytMQLUPx3MKqE_XIXwuXfB}{Link}
\\ \hline
Larraondo Lancho, Alejandro Jesús &
3 &
4 &
\href{https://drive.google.com/open?id=1a_VwNehS8UytMQLUPx3MKqE_XIXwuXfB}{Link}
\\ \hline
Mamani Chirinos, Luis &
2 &
1 &
\href{https://drive.google.com/open?id=1a_VwNehS8UytMQLUPx3MKqE_XIXwuXfB}{Link}
\\ \hline
Mendoza Villarroel, Alexis &
4 &
4 &
\href{https://drive.google.com/open?id=1a_VwNehS8UytMQLUPx3MKqE_XIXwuXfB}{Link}
\\ \hline
Quincho Mamani, Lehi &
3 &
4 &
\href{https://drive.google.com/open?id=1a_VwNehS8UytMQLUPx3MKqE_XIXwuXfB}{Link}
\\ \hline
Quispe Quicano, Julio Cesar &
3 &
4 &
\href{https://drive.google.com/open?id=1a_VwNehS8UytMQLUPx3MKqE_XIXwuXfB}{Link}
\\ \hline
Turpo Apaza, Crhistian Andrew &
4 &
4 &
\href{https://drive.google.com/open?id=1a_VwNehS8UytMQLUPx3MKqE_XIXwuXfB}{Link}
\\ \hline
Uñapilco Chambi, Katherine &
4 &
4 &
\href{https://drive.google.com/open?id=1a_VwNehS8UytMQLUPx3MKqE_XIXwuXfB}{Link}
\\ \hline
\end{tabular}
\caption{Nivel del Logro para Cada Estudiante en el Resultado del Estudiante \lbrack c\rbrack}
\label{tab:nivel_estudiante_c} 
\end{table}

\newpage

\begin{table}[h]
\centering
\begin{tabular}{C{2cm}|C{2.5cm}|C{2.5cm}|C{2.5cm}|C{2.5cm}|C{4cm}}
\hline
\textbf{Indicador} & 
\textbf{Nivel 1 Insatisfactorio} & 
\textbf{Nivel 2 Parcialmente Satisfactorio} & 
\textbf{Nivel 3 Satisfactorio} & 
\textbf{Nivel 4 Excelente} &
\textbf{Nivel del Logro 2019A} 
\\ \hline
c1 &
\makecell{0 Estudiante(s) \\ (0.00\%)} &
\makecell{2 Estudiante(s) \\ (12.50\%)} &
\makecell{5 Estudiante(s) \\ (31.25\%)} &
\makecell{9 Estudiante(s) \\ (56.25\%)} &
\makecell{14 Estudiante(s) \\ (87.75\%) \\ > Objetivo (60\%)}
\\ \hline
c2 &
\makecell{1 Estudiante(s) \\ (6.25\%)} &
\makecell{1 Estudiante(s) \\ (6.25\%)} &
\makecell{2 Estudiante(s) \\ (12.50\%)} &
\makecell{12 Estudiante(s) \\ (75.50\%)} &
\makecell{14 Estudiante(s) \\ (87.50\%) \\ > Objetivo (60\%)}
\\ \hline
\end{tabular}
\caption{Resumen de los Niveles del Logro en el Resultado del Estudiante \lbrack c\rbrack}
\label{tab:nivel_resumen_c}
\end{table}

\end{landscape}

\begin{figure}
\centering
\begin{tikzpicture}[scale=1.00]
\pie [rotate = 180, 
	color={orange,yellow,green}]
    {12.50/Parcialmente Satisfactorio,
     31.25/Satisfactorio,
     56.25/Excelente}
\end{tikzpicture}
\caption{Resumen de los Niveles del Logro en el Resultado del Estudiante \lbrack c1\rbrack}
\end{figure}

\begin{figure}
\centering
\begin{tikzpicture}[scale=1.00]
\pie [rotate = 180, 
	color={red,orange,yellow,green}]
    {6.25/Insatisfactorio,
     6.25/Parcialmente Satisfactorio,
     12.50/Satisfactorio,
     75.50/Excelente}
\end{tikzpicture}
\caption{Resumen de los Niveles del Logro en el Resultado del Estudiante \lbrack c2\rbrack}
\end{figure}

%------------------------------------------------ Fin de RE c

%------------------ RE d

\begin{landscape}
\subsection{Resultado del Estudiante \lbrack d\rbrack:}
Trabajo Individual y en Equipo: La capacidad de desenvolverse eficazmente como individuo, como miembro o líder de equipos diversos. \textbf{Nivel 1 (Comprende)}

\begin{table}[h]
\centering
\begin{tabular}{C{5cm}|C{2cm}|C{2cm}|C{5cm}|C{2cm}|C{2cm}|C{2cm}}
\hline
\textbf{Indicador de Desempeño} & 
\textbf{Curso} & 
\textbf{Métodos - Assessment} & 
\textbf{Fuentes de Assessment (curso, semana y actividad)} & 
\textbf{Ciclo de Assessment} & 
\textbf{Coordinador Assesment} & 
\textbf{Nivel de Logro Esperado}
\\ \hline
d1. Realiza la implementación de un Algoritmo Bioinspirado no visto en el Curso &
Computación Bioinspirada &
Rúbrica (Directa) &
\makecell{Computación Bioinspirada \\ Semana 17 \\ Trabajo Final} &
2019-A &
Edward Hinojosa C. &
60\% 
\\ \hline
d2. Muestra ejemplos de la resolución de problemas complejos de un Algoritmo Bioinspirado no visto en el Curso &
Computación Bioinspirada &
Rúbrica (Directa) &
\makecell{Computación Bioinspirada \\ Semana 17 \\ Trabajo Final} &
2019-A &
Edward Hinojosa C. &
60\%
\\ \hline
\end{tabular}
\caption{Indicadores de Desempeño del Resultado del Estudiante \lbrack d\rbrack}
\label{tab:nivel_d}
\end{table}

\newpage

\begin{table}[h]
\centering
\begin{tabular}{C{5cm}|C{4cm}|C{4cm}|C{4cm}|C{4cm}}
\hline
\textbf{Indicador de Desempeño} & 
\textbf{1. Insatisfactorio} & 
\textbf{2. Parcialmente Satisfactorio} & 
\textbf{3. Satisfactorio} & 
\textbf{4. Excelente} 
\\ \hline
d1. Realiza la implementación de un Algoritmo Bioinspirado no estudiado en el Curso &
No realiza la implementación de un Algortimo Bioinspirado &
Realiza con errores la implementaición de un Algoritmo Bioinspirado &
Realiza sin errores la implementación de un Algoritmo Bioinspirado &
Realiza sin errores la implementación de un Algoritmo Bioinspirado y optimizar su rendimiento
\\ \hline
d2. Muestra ejemplos de la resolución de problemas complejos usando un Algoritmo Bioinspirado no estudiado en el Curso &
No muestra ejemplos de la resolución de problemas complejos usando un Algoritmo Bioinspirado &
Muestra un ejemplo de la resolución de problemas complejos usando un Algoritmo Bioinspirado &
Muestra dos ejemplos de la resolución de problemas complejos usando un Algoritmo Bioinspirado &
Muestra tres o más ejemplos de la resolución de problemas complejos usando un Algoritmo Bioinspirado
\\ \hline
\end{tabular}
\caption{Rúbrica a Usarse para cada Nivel del Logro en el Resultado del Estudiante \lbrack d\rbrack}
\label{tab:nivel_rubrica_d} 
\end{table}

\newpage

\begin{table}[h]
\centering
\begin{tabular}{L{6cm}|C{1cm}|C{1cm}|C{2cm}}
\hline
\textbf{Nombre} & 
\textbf{d1} & 
\textbf{d2} & 
\textbf{Evidencia} 
\\ \hline
Amable Romero, Diego Javier &
4 &
2 &
\href{https://drive.google.com/open?id=1RGE-0KmemYyM1_d2kqPFSByuEEYTf70C}{Link}
\\ \hline
Bernal Chahuayo, Luis Antonio &
4 &
2 &
\href{https://drive.google.com/open?id=1FeRY8FPwENPs6JGo3LrUYZigr4FfQQH4}{Link}
\\ \hline
Caceres Zegarra, Luis Gustavo &
3 &
2 &
\href{https://drive.google.com/open?id=1aRhn2lBo3sZXlTMA4sh9wJ5r5g2bXE8T}{Link}
\\ \hline
Espinel Quispe, Ingrid Sally &
1 &
1 &
\href{https://drive.google.com/open?id=135na5wEvQ_LrngLGUhVwp92BpkZ2PblH}{Link}
\\ \hline
Gordillo Viña, Karen &
4 &
4 &
\href{https://drive.google.com/open?id=12-LqWuAXj5IB53-Me7gi4mJz6In7qewI}{Link}
\\ \hline
Gutierrez Salazar, Enrique Alonzo &
4 &
3 &
\href{https://drive.google.com/open?id=10fti0Nvn_GhppfJQ0cGxxPrEwGgbgulw}{Link}
\\ \hline
Hancco Tancayllo, Hermith &
2 &
2 &
\href{https://drive.google.com/open?id=15XyNJkvHo4XuC1ylerS8IQey5eImNxRR}{Link}
\\ \hline
Huaman Canqui Jair, Francesco &
3 &
2 &
\href{https://drive.google.com/open?id=1KHvPDBSUBZqywpxbN_niUf8PdkRjzBoC}{Link}
\\ \hline
Lacuaña Apaza, Margarita &
3 &
2 &
\href{https://drive.google.com/open?id=1OZAFwkOkPnFwLvVwAAH-jvAhziwBx09o}{Link}
\\ \hline
Larraondo Lancho, Alejandro Jesús &
4 &
3 &
\href{https://drive.google.com/open?id=1qjRtWRIrA7fcinCeWOX74rG9vonaawIF}{Link}
\\ \hline
Mamani Chirinos, Luis &
2 &
2 &
\href{https://drive.google.com/open?id=1JrGsIkJzS_lwTkOaSZdIipVaRmakLodP}{Link}
\\ \hline
Mendoza Villarroel, Alexis &
4 &
3 &
\href{https://drive.google.com/open?id=1xQ0p9bxbJRba3aqW5BiKMy3jFEyogEPM}{Link}
\\ \hline
Quincho Mamani, Lehi &
4 &
4 &
\href{https://drive.google.com/open?id=1X4sgTvGoHOD-xrGULO0eLx6MiX9UB26f}{Link}
\\ \hline
Quispe Quicano, Julio Cesar &
3 &
3 &
\href{https://drive.google.com/open?id=1498VuZAYY9B6AScuvHA5EMFrCAxscPer}{Link}
\\ \hline
Turpo Apaza, Crhistian Andrew &
3 &
3 &
\href{https://drive.google.com/open?id=14HnFWBEkh2y6LEVVtxKpaJqsddbscORE}{Link}
\\ \hline
Uñapilco Chambi, Katherine &
3 &
4 &
\href{https://drive.google.com/open?id=1fXM-Ufvhy0M0FVDEkZ8dclLG2OK0P8ga}{Link}
\\ \hline
\end{tabular}
\caption{Nivel del Logro para Cada Estudiante en el Resultado del Estudiante \lbrack d\rbrack}
\label{tab:nivel_estudiante_d} 
\end{table}

\newpage

\begin{table}[h]
\centering
\begin{tabular}{C{2cm}|C{2.5cm}|C{2.5cm}|C{2.5cm}|C{2.5cm}|C{4cm}}
\hline
\textbf{Indicador} & 
\textbf{Nivel 1 Insatisfactorio} & 
\textbf{Nivel 2 Parcialmente Satisfactorio} & 
\textbf{Nivel 3 Satisfactorio} & 
\textbf{Nivel 4 Excelente} &
\textbf{Nivel del Logro 2019A} 
\\ \hline
d1 &
\makecell{1 Estudiante(s) \\ (6.25\%)} &
\makecell{2 Estudiante(s) \\ (12.50\%)} &
\makecell{6 Estudiante(s) \\ (37.50\%)} &
\makecell{7 Estudiante(s) \\ (43.75\%)} &
\makecell{13 Estudiante(s) \\ (81.25\%) \\ > Objetivo (60\%)}
\\ \hline
d2 &
\makecell{1 Estudiante(s) \\ (6.25\%)} &
\makecell{7 Estudiante(s) \\ (43.75\%)} &
\makecell{5 Estudiante(s) \\ (31.25\%)} &
\makecell{3 Estudiante(s) \\ (18.75\%)} &
\makecell{8 Estudiante(s) \\ (50.00\%) \\ < Objetivo (60\%)}
\\ \hline
\end{tabular}
\caption{Resumen de los Niveles del Logro en el Resultado del Estudiante \lbrack d\rbrack}
\label{tab:nivel_resumen_d}
\end{table}

\end{landscape}

\begin{figure}
\centering
\begin{tikzpicture}[scale=1.00]
\pie [rotate = 180, 
	color={red,orange,yellow,green}]
    {6.25/Insatisfactorio,
     12.50/Parcialmente Satisfactorio,
     37.50/Satisfactorio,
     43.75/Excelente}
\end{tikzpicture}
\caption{Resumen de los Niveles del Logro en el Resultado del Estudiante \lbrack d1\rbrack}
\end{figure}

\begin{figure}
\centering
\begin{tikzpicture}[scale=1.00]
\pie [rotate = 180, 
	color={red,orange,yellow,green}]
    {6.25/Insatisfactorio,
     43.75/Parcialmente Satisfactorio,
     31.25/Satisfactorio,
     18.75/Excelente}
\end{tikzpicture}
\caption{Resumen de los Niveles del Logro en el Resultado del Estudiante \lbrack d2\rbrack}
\end{figure}

%------------------------------------------------ Fin de RE d

%------------------ RE e
\begin{landscape}
\subsection{Resultado del Estudiante \lbrack e\rbrack:}
Comunicación: La capacidad de comunicarse eficazmente, de forma oral y escrita, en una variedad de contextos profesionales. \textbf{Nivel 1 (Comprende)}

\begin{table}[h]
\centering
\begin{tabular}{C{5cm}|C{2cm}|C{2cm}|C{5cm}|C{2cm}|C{2cm}|C{2cm}}
\hline
\textbf{Indicador de Desempeño} & 
\textbf{Curso} & 
\textbf{Métodos - Assessment} & 
\textbf{Fuentes de Assessment (curso, semana y actividad)} & 
\textbf{Ciclo de Assessment} & 
\textbf{Coordinador Assesment} & 
\textbf{Nivel de Logro Esperado}
\\ \hline
e1. Realiza la exposición de su Trabajo Final &
Computación Bioinspirada &
Rúbrica (Directa) &
\makecell{Computación Bioinspirada \\ Semana 17 \\ Trabajo Final} &
2019-A &
Edward Hinojosa C. &
60\% 
\\ \hline
e2. Realiza un video de explicación de sus Trabajo Final &
Computación Bioinspirada &
Rúbrica (Directa) &
\makecell{Computación Bioinspirada \\ Semana 17 \\ Trabajo Final} &
2019-A &
Edward Hinojosa C. &
60\%
\\ \hline
\end{tabular}
\caption{Indicadores de Desempeño del Resultado del Estudiante \lbrack e\rbrack}
\label{tab:nivel_e}
\end{table}

\newpage

\begin{table}[h]
\centering
\begin{tabular}{C{5cm}|C{4cm}|C{4cm}|C{4cm}|C{4cm}}
\hline
\textbf{Indicador de Desempeño} & 
\textbf{1. Insatisfactorio} & 
\textbf{2. Parcialmente Satisfactorio} & 
\textbf{3. Satisfactorio} & 
\textbf{4. Excelente} 
\\ \hline
e1. Realiza la exposición de su Trabajo Final &
No realiza la exposición de su Trabajo Final &
Realiza la exposición de su trabajo final sin preparación &
Realiza la exposición de su trabajo final con preparación &
Realiza la exposición de su trabajo final sin preparación y de forma detallada
\\ \hline
e2. Realiza un video de explicación de sus Trabajo Final &
No realiza el video de explicación de su Trabajo Final &
Realiza el video de explicación de su Trabajo Final sin preparación &
Realiza el video de explicación de su Trabajo Final con preparación &
Realiza el video de explicación de su Trabajo Final con preparación y de forma detallada
\\ \hline
\end{tabular}
\caption{Rúbrica a Usarse para cada Nivel del Logro en el Resultado del Estudiante \lbrack e\rbrack}
\label{tab:nivel_rubrica_e} 
\end{table}

\newpage

\begin{table}[h]
\centering
\begin{tabular}{L{6cm}|C{1cm}|C{1cm}|C{2cm}}
\hline
\textbf{Nombre} & 
\textbf{e1} & 
\textbf{e2} & 
\textbf{Evidencia} 
\\ \hline
Amable Romero, Diego Javier &
3 &
4 &
\href{https://drive.google.com/open?id=1RGE-0KmemYyM1_d2kqPFSByuEEYTf70C}{Link}
\\ \hline
Bernal Chahuayo, Luis Antonio &
3 &
2 &
\href{https://drive.google.com/open?id=1FeRY8FPwENPs6JGo3LrUYZigr4FfQQH4}{Link}
\\ \hline
Caceres Zegarra, Luis Gustavo &
2 &
3 &
\href{https://drive.google.com/open?id=1aRhn2lBo3sZXlTMA4sh9wJ5r5g2bXE8T}{Link}
\\ \hline
Espinel Quispe, Ingrid Sally &
2 &
2 &
\href{https://drive.google.com/open?id=135na5wEvQ_LrngLGUhVwp92BpkZ2PblH}{Link}
\\ \hline
Gordillo Viña, Karen &
3 &
3 &
\href{https://drive.google.com/open?id=12-LqWuAXj5IB53-Me7gi4mJz6In7qewI}{Link}
\\ \hline
Gutierrez Salazar, Enrique Alonzo &
2 &
3 &
\href{https://drive.google.com/open?id=10fti0Nvn_GhppfJQ0cGxxPrEwGgbgulw}{Link}
\\ \hline
Hancco Tancayllo, Hermith &
3 &
3 &
\href{https://drive.google.com/open?id=15XyNJkvHo4XuC1ylerS8IQey5eImNxRR}{Link}
\\ \hline
Huaman Canqui Jair, Francesco &
3 &
3 &
\href{https://drive.google.com/open?id=1KHvPDBSUBZqywpxbN_niUf8PdkRjzBoC}{Link}
\\ \hline
Lacuaña Apaza, Margarita &
3 &
2 &
\href{https://drive.google.com/open?id=1OZAFwkOkPnFwLvVwAAH-jvAhziwBx09o}{Link}
\\ \hline
Larraondo Lancho, Alejandro Jesús &
3 &
4 &
\href{https://drive.google.com/open?id=1qjRtWRIrA7fcinCeWOX74rG9vonaawIF}{Link}
\\ \hline
Mamani Chirinos, Luis &
3 &
2 &
\href{https://drive.google.com/open?id=1JrGsIkJzS_lwTkOaSZdIipVaRmakLodP}{Link}
\\ \hline
Mendoza Villarroel, Alexis &
3 &
2 &
\href{https://drive.google.com/open?id=1xQ0p9bxbJRba3aqW5BiKMy3jFEyogEPM}{Link}
\\ \hline
Quincho Mamani, Lehi &
3 &
2 &
\href{https://drive.google.com/open?id=1X4sgTvGoHOD-xrGULO0eLx6MiX9UB26f}{Link}
\\ \hline
Quispe Quicano, Julio Cesar &
3 &
2 &
\href{https://drive.google.com/open?id=1498VuZAYY9B6AScuvHA5EMFrCAxscPer}{Link}
\\ \hline
Turpo Apaza, Crhistian Andrew &
3 &
2 &
\href{https://drive.google.com/open?id=14HnFWBEkh2y6LEVVtxKpaJqsddbscORE}{Link}
\\ \hline
Uñapilco Chambi, Katherine &
3 &
4 &
\href{https://drive.google.com/open?id=1fXM-Ufvhy0M0FVDEkZ8dclLG2OK0P8ga}{Link}
\\ \hline
\end{tabular}
\caption{Nivel del Logro para Cada Estudiante en el Resultado del Estudiante \lbrack e\rbrack}
\label{tab:nivel_estudiante_e} 
\end{table}

\newpage

\begin{table}[h]
\centering
\begin{tabular}{C{2cm}|C{2.5cm}|C{2.5cm}|C{2.5cm}|C{2.5cm}|C{4cm}}
\hline
\textbf{Indicador} & 
\textbf{Nivel 1 Insatisfactorio} & 
\textbf{Nivel 2 Parcialmente Satisfactorio} & 
\textbf{Nivel 3 Satisfactorio} & 
\textbf{Nivel 4 Excelente} &
\textbf{Nivel del Logro 2019A} 
\\ \hline
e1 &
\makecell{0 Estudiante(s) \\ (0.00\%)} &
\makecell{3 Estudiante(s) \\ (18.75\%)} &
\makecell{13 Estudiante(s) \\ (81.25\%)} &
\makecell{0 Estudiante(s) \\ (0.00\%)} &
\makecell{13 Estudiante(s) \\ (81.25\%) \\ > Objetivo (60\%)}
\\ \hline
e2 &
\makecell{0 Estudiante(s) \\ (0.00\%)} &
\makecell{8 Estudiante(s) \\ (50.00\%)} &
\makecell{5 Estudiante(s) \\ (31.25\%)} &
\makecell{3 Estudiante(s) \\ (18.75\%)} &
\makecell{8 Estudiante(s) \\ (50.00\%) \\ < Objetivo (60\%)}
\\ \hline
\end{tabular}
\caption{Resumen de los Niveles del Logro en el Resultado del Estudiante \lbrack e\rbrack}
\label{tab:nivel_resumen_e}
\end{table}

\end{landscape}

\begin{figure}
\centering
\begin{tikzpicture}[scale=1.00]
\pie [rotate = 180, 
	color={orange,yellow}]
    {18.75/Parcialmente Satisfactorio,
     81.25/Satisfactorio}
\end{tikzpicture}
\caption{Resumen de los Niveles del Logro en el Resultado del Estudiante \lbrack e1\rbrack}
\end{figure}

\begin{figure}
\centering
\begin{tikzpicture}[scale=1.00]
\pie [rotate = 180, 
	color={orange,yellow,green}]
    {50.00/Parcialmente Satisfactorio,
     31.25/Satisfactorio,
     18.75/Excelente}
\end{tikzpicture}
\caption{Resumen de los Niveles del Logro en el Resultado del Estudiante \lbrack e2\rbrack}
\end{figure}

%------------------------------------------------ Fin de RE e