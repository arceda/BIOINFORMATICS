\documentclass{article}
\usepackage[utf8]{inputenc}
\usepackage[top=3cm, bottom=3cm, outer=3cm, inner=3cm]{geometry}
\usepackage{graphicx}
\usepackage{url}
\usepackage{cite}
\usepackage{hyperref}
\usepackage{array}
\usepackage{multicol}
\newcolumntype{x}[1]{>{\centering\arraybackslash\hspace{0pt}}p{#1}}
\usepackage[ruled,vlined]{algorithm2e}

%%%%%%%%%%%%%%%%%%%%%%%%%%%%%%%%%%%%%%%%%%%%%%%%%%%%%%%%%%%%%%%%%%%%%%%%%%%%
%%%%%%%%%%%%%%%%%%%%%%%%%%%%%%%%%%%%%%%%%%%%%%%%%%%%%%%%%%%%%%%%%%%%%%%%%%%%
\newcommand{\csemail}{vmachacaa@unsa.edu.pe}
\newcommand{\csdocente}{MSc. Vicente Machaca Arceda}
\newcommand{\cscurso}{Bioinformática}
\newcommand{\csuniversidad}{Universidad Nacional de San Agustín de Arequipa}
\newcommand{\csescuela}{Escuela Profesional de Ciencia de la Computación}
\newcommand{\cspracnr}{}
\newcommand{\cstema}{-}
%%%%%%%%%%%%%%%%%%%%%%%%%%%%%%%%%%%%%%%%%%%%%%%%%%%%%%%%%%%%%%%%%%%%%%%%%%%%
%%%%%%%%%%%%%%%%%%%%%%%%%%%%%%%%%%%%%%%%%%%%%%%%%%%%%%%%%%%%%%%%%%%%%%%%%%%%

\usepackage{fancyhdr}
\pagestyle{fancy}
\fancyhf{}
\setlength{\headheight}{30pt}
\renewcommand{\headrulewidth}{1pt}
\renewcommand{\footrulewidth}{1pt}
\fancyhead[L]{ \raisebox{0.1\height}{\includegraphics[width=3cm]{img/logo_unsa}} }
\fancyhead[C]{ \fontsize{7}{7}\selectfont	\csuniversidad \\ \csescuela \\ \textbf{\cscurso} \\ \raisebox{\height}{ } }
\fancyhead[R]{ \raisebox{0.1\height}{\includegraphics[width=1.5cm]{img/logo_epcc_unsa}} }
\fancyfoot[L]{MSc. Vicente Machaca}
\fancyfoot[C]{\cscurso}
\fancyfoot[R]{Página \thepage}


\usepackage[english,spanish]{babel}
\AtBeginDocument{\selectlanguage{spanish}}
\renewcommand{\figurename}{Figura}
\renewcommand{\refname}{Referencias}
\renewcommand{\tablename}{Tabla}


% para el codigo fuente
\usepackage{listings}
\usepackage{color}
\definecolor{dkgreen}{rgb}{0,0.6,0}
\definecolor{gray}{rgb}{0.5,0.5,0.5}
\definecolor{mauve}{rgb}{0.58,0,0.82}
\lstset{frame=tb,
	language=Python,
	aboveskip=3mm,
	belowskip=3mm,
	showstringspaces=false,
	columns=flexible,
	basicstyle={\small\ttfamily},
	numbers=none,
	numberstyle=\tiny\color{gray},
	keywordstyle=\color{blue},
	commentstyle=\color{dkgreen},
	stringstyle=\color{mauve},
	breaklines=true,
	breakatwhitespace=true,
	tabsize=3
}

\usepackage[spanish]{babel}
\AtBeginDocument{\selectlanguage{spanish}}
\renewcommand{\figurename}{Figura}
\renewcommand{\refname}{Referencias}
\renewcommand{\tablename}{Tabla}




\begin{document}
	
	% image background
	%%%%%%%%%%%%%%%%%%%%%%%%%%%%%%%%%%%%%%%%%%%%%%%%%%%%%%%%%%%%%%%%%%%%%%%%%%	
	%%%%%%%%%%%%%%%%%%%%%%%%%%%%%%%%%%%%%%%%%%%%%%%%%%%%%%%%%%%%%%%%%%%%%%%%%%	
	%\AddToShipoutPicture{\BackgroundPic}
	%\AddToShipoutPicture*{\BackgroundPic} %solo laprimera página
	%%%%%%%%%%%%%%%%%%%%%%%%%%%%%%%%%%%%%%%%%%%%%%%%%%%%%%%%%%%%%%%%%%%%%%%%%%	
	%%%%%%%%%%%%%%%%%%%%%%%%%%%%%%%%%%%%%%%%%%%%%%%%%%%%%%%%%%%%%%%%%%%%%%%%%%	
	
	
	\begin{center}	
		\fontsize{15}{15} \textbf{Examen de entrada} \\ \vspace{0.25cm}
		\csdocente \\
		\today \\ \vspace{0.25cm}
	\end{center}

	\begin{flushleft}
	Nombre: \\
	Apellidos:\\
	CUI:\\
	\end{flushleft}
	
	%\begin{table}[h]
	%	\begin{tabular}{|x{4.7cm}|x{4.8cm}|x{4.8cm}|}
	%		\hline 
	%		\textbf{DOCENTE} & \textbf{CARRERA}  & \textbf{CURSO}   \\
	%		\hline 
	%		\csdocente & \csescuela & \cscurso    \\
	%		\hline 
	%	\end{tabular}
	%\end{table}
	
	%\begin{table}[h]
	%	\begin{tabular}{|x{5cm}|x{6cm}|x{5cm}|}
	%		\hline 
	%		\textbf{PRÁCTICA} & \textbf{TEMA}  & \textbf{DURACIÓN}   \\
	%		\hline 
	%		\cspracnr & \cstema & 3 horas   \\
	%		\hline 
	%	\end{tabular}
	%\end{table}
	
	

	%\section{Actividades}
	%En esta ocasión vamos a descargar dos secuencias de proteinas y aplicaremos el algoritmo de Dot matrix.
	
	
	
	
	\section*{Preguntas}
	
	\begin{enumerate}
		\item Explique qué es el ADN. \textbf{(4 puntos)}
		\item Explique qué son los genes y proteínas. \textbf{(4 puntos)}
		\item ¿A que cree que se deban las mutaciones en el ADN?. \textbf{(4 puntos)}
		\item Explique como es el proceso de aprendizaje de los modelos de \textit{machine learning}. \textbf{(4 puntos)}
		\item Implementar un programa en el lenguaje de su preferencia que reciba como entrada dos cadenas de texto y retorne un valor numérico indicando el grado de similitud entre dichas cadenas. Usted puede definir qué criterios tomar para retornar el grado de similitud. \textbf{(4 puntos)}\\
		
		Ejemplos de cadenas similares: 
		\begin{itemize}
			\item 	cadena\_1: ACGT
			\item 	cadena\_2: ACGGT \\
			\item 	cadena\_1: GTAACGT
			\item 	cadena\_2: GTAAGT \\
			\item 	cadena\_1: ACGT
			\item 	cadena\_2: AGGT
		\end{itemize}		
		
	\end{enumerate}
	

	
	%\clearpage
	%\bibliographystyle{ieeetr}
	%\bibliography{bibliography}
	
\end{document}
