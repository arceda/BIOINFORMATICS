\documentclass{article}
\usepackage[utf8]{inputenc}
\usepackage[top=3cm, bottom=3cm, outer=3cm, inner=3cm]{geometry}
\usepackage{graphicx}
\usepackage{url}
\usepackage{cite}
\usepackage{hyperref}
\usepackage{array}
\usepackage{float}
\usepackage{multicol}
\usepackage{amsmath, xparse}
\newcolumntype{x}[1]{>{\centering\arraybackslash\hspace{0pt}}p{#1}}

%%%%%%%%%%%%%%%%%%%%%%%%%%%%%%%%%%%%%%%%%%%%%%%%%%%%%%%%%%%%%%%%%%%%%%%%%%%%
%%%%%%%%%%%%%%%%%%%%%%%%%%%%%%%%%%%%%%%%%%%%%%%%%%%%%%%%%%%%%%%%%%%%%%%%%%%%
\newcommand{\csemail}{vmachacaa@unsa.edu.pe}
\newcommand{\csdocente}{MSc. Vicente Machaca Arceda}
\newcommand{\cscurso}{Bioinformática}
\newcommand{\csuniversidad}{Universidad Nacional de San Agustín de Arequipa}
\newcommand{\csescuela}{Escuela Profesional de Ciencia de la Computación}
\newcommand{\cspracnr}{-}
\newcommand{\cstema}{Simulación de fluidos}
%%%%%%%%%%%%%%%%%%%%%%%%%%%%%%%%%%%%%%%%%%%%%%%%%%%%%%%%%%%%%%%%%%%%%%%%%%%%
%%%%%%%%%%%%%%%%%%%%%%%%%%%%%%%%%%%%%%%%%%%%%%%%%%%%%%%%%%%%%%%%%%%%%%%%%%%%

\usepackage{fancyhdr}
\pagestyle{fancy}
\fancyhf{}
\setlength{\headheight}{30pt}
\renewcommand{\headrulewidth}{1pt}
\renewcommand{\footrulewidth}{1pt}
\fancyhead[L]{ \raisebox{0.1\height}{\includegraphics[width=3cm]{img/logo_unsa}} }
\fancyhead[C]{ \fontsize{7}{7}\selectfont	\csuniversidad \\ \csescuela \\ \textbf{\cscurso} \\ \raisebox{\height}{ } }
\fancyhead[R]{ \raisebox{0.1\height}{\includegraphics[width=1.5cm]{img/logo_epcc_unsa}} }
\fancyfoot[L]{MSc. Vicente Machaca}
\fancyfoot[C]{\cscurso}
\fancyfoot[R]{Página \thepage}

\usepackage{listings}
\usepackage{color}
\definecolor{lightgray}{rgb}{.9,.9,.9}
\definecolor{darkgray}{rgb}{.4,.4,.4}
\definecolor{purple}{rgb}{0.65, 0.12, 0.82}
\definecolor{forestgreen}{rgb}{0.12, 0.6, 0.32}
\definecolor{blue}{rgb}{0, 0,1}

\lstdefinelanguage{JavaScript}{
	keywords={typeof, new, true, false, catch, function, return, null, catch, switch, var, if, in, while, do, else, case, break, for, let, const},
	keywordstyle=\color{blue}\bfseries,
	ndkeywords={class, export, boolean, throw, implements, import, this},
	ndkeywordstyle=\color{blue}\bfseries,
	identifierstyle=\color{black},
	sensitive=false,
	comment=[l]{//},
	morecomment=[s]{/*}{*/},
	commentstyle=\color{forestgreen}\ttfamily,
	stringstyle=\color{red}\ttfamily,
	morestring=[b]',
	morestring=[b]"
}

\lstset{frame=tb,
	language=JavaScript,
	%backgroundcolor=\color{white},
	extendedchars=true,
	basicstyle=\footnotesize\ttfamily,
	showstringspaces=false,
	showspaces=false,
	numbers=none,
	numberstyle=\footnotesize,
	numbersep=9pt,
	tabsize=2,
	breaklines=true,
	showtabs=false,
	captionpos=b
}

\usepackage[english,spanish]{babel}
\AtBeginDocument{\selectlanguage{spanish}}
\renewcommand{\figurename}{Figura}
\renewcommand{\refname}{Referencias}
\renewcommand{\tablename}{Tabla}


% para la imagen de fondo
%\usepackage{eso-pic}
%\newcommand\BackgroundPic{%
%\put(0,0){%
%\parbox[b][\paperheight]{\paperwidth}{%
%\vfill
%\centering
%\includegraphics[width=\paperwidth,height=\paperheight]{../img/background4.png}%
%\vfill
%}}}


%\title{\textbf{Lista de proyectos}}
%\author{\csdocente}
%\date{\today}



\begin{document}
		
	%\maketitle
	
	\begin{center}	
		\fontsize{15}{15} \textbf{Segundo Examen Parcial} %\\ \vspace{0.25cm}
		%\csdocente \\
	%	\today \\ \vspace{0.25cm}
	\end{center}

\begin{table}[h]
	\begin{tabular}{|x{4.7cm}|x{4.8cm}|x{4.8cm}|}
		\hline 
		\textbf{DOCENTE} & \textbf{CARRERA}  & \textbf{CURSO}   \\
		\hline 
		\csdocente & \csescuela & \cscurso    \\
		\hline 
	\end{tabular}
\end{table}
	

\section{Resultados del estudiante}
\begin{itemize}
	\item (a) Conocimientos en computación
	\item (b) Análisis de problemas.
	\item (c) Diseño y desarrollo de soluciones.
	\item (d) Trabajo individual y en equipo.
	\item (h) Uso de herramientas modernas.
\end{itemize}


\section{Competencias del trabajo}
\begin{itemize}
	\item Investiga sobre algoritmos y conceptos de Bioinformática para el alineamiento de secuencias y generación de árboles filogenéticos (a, h).
	\item Implementa una aplicación Web para el alineamiento de secuencias y generación de árboles filogenéticos (b, c, d).
\end{itemize}

\section{Equipos y materiales}
\begin{itemize}
	\item C++, Python, Javascript
	\item BioPython
\end{itemize}

\section{Entregables}
\begin{itemize}
	\item Se debe elaborar un informe en \textit{Latex} donde se explique el proyecto.
	\item En el informe se debe agregar un enlace al repositorio Github.
\end{itemize}




\clearpage

%\section{Actividades}
%En esta ocasión vamos a descargar dos secuencias de proteinas y aplicaremos el algoritmo de Dot matrix.




\section{Descripción del trabajo}

Implementar una aplicación Web, donde se pueda realizar el alineamiento de secuencias y generación de árboles filogenéticos. La aplicación sigue el siguiente proceso: (1) debe alinear las secuencias, (2) generar las distancias y (3) luego mostrar el árbol filogenético. Para esto la aplicación debe tener implementado estos algoritmos:
	\begin{itemize}
		\item Alineamiento global (Needleman–Wunsch).
		\item Alineamiento local (Smith-waterman).
		\item BLAST.
		\item Alineamiento multiple MUSCLE o CLUSTAL (puede utilizar una librería).		
		\item \textit{Jukes-cantor model} (puede utilizar una librería).
		\item \textit{Kimura model} (puede utilizar una librería).
		\item UPGMA.
		\item \textit{Neighbor Joining.}
	\end{itemize}

Puede adicionar otros algoritmos según vea conveniente. Los algoritmos adicionales pueden ser utilizados con ayuda de librerías. Luego, tambien es importante alinear, procesar distancias y construir el árbol filogenético de manera aislada. Por ejemplo, el usuario podría solo querer construir el árbol filogenético, ingresando como entrada un \textit{csv} con las distancias entre secuencias.\\

El informe debe tener un enlace a un repositorío, en este se evalurá la participación de cada integrante. Si se detecta que no hay trabajo en equipo, todo el grupo tendra puntos en contra.

\section{Rúbricas}

\begin{table}[H]
	\begin{tabular}{|p{6cm}|x{2.5cm}|x{3cm}|x{2.5cm}|}
		\hline 
		\textbf{Rúbrica} & \textbf{Cumple}  & \textbf{Cumple con obs.}  & \textbf{No cumple} \\
		\hline 
		\textbf{Informe}: El informe debe estar en Latex, tiene un contenido detallado y un formato limpio y facil de leer (b, c).   & 2 & 1 & 0   \\
		\hline 
		\textbf{Trabajo en equipo}: Se comprueba el trabajo en equipo en el repositorio Github (d).  & 4 & 2 & 0   \\
		\hline 
		\textbf{Implementación}: La aplicación Web tiene todas las funcionalidades requeridas (b, c).  & 9 & 4.5 & 0   \\		\hline 	
		\textbf{Métodos adicionales}: La aplicación tiene funcionalidadeds adicionales u otro métodos a los ya requeridos (h).  & 3 & 1.5 & 0   \\		\hline 	
		\textbf{Presentación}: El alumno demuestra dominio del tema durante la presentación (a).  & 2 & 1 & 0   \\		\hline 		
	\end{tabular}
\end{table}
	
	
%\clearpage

%\bibliographystyle{apalike}
%\bibliographystyle{IEEEtranN}
%\bibliography{bibliography}	
	
\end{document}