\chapter{Mapeo del Resultado del Estudiante}
\newpage

\pagestyle{fancy} 

% Para agregar PDF solo debe eliminar el porcentaje (%) y cambiar el nombre del archivo .pdf. 
%\includepdf[pages={1-}]{pdfs/4_mapeo_estudiante.pdf}

%-------------------------------------------------------------------------------
%-------------------------------------------------------------------------------

\clearpage
\begin{table}[H]
	\centering
	\begin{tabular}{|x{\textwidth}|}
	\hline 
	\MakeUppercase{\csuniversidad} \\
	\\
	\begin{minipage}{.3\textwidth}
		\includegraphics[width=\textwidth]{../imgs/cs2}
	\end{minipage} \\
	\\
	\textbf{\MakeUppercase{\csepcc}} \\
	\MakeUppercase{\csfacultad} \\
	\MakeUppercase{\csdepartamento} \\
	\hline 
	\end{tabular}
\end{table}

\vspace{1cm}
{\centering 
	\MakeUppercase{\textbf{Asignatura}}: \MakeUppercase{\cscourse}} \\
\vspace{0.5cm}

\begin{table}[H]
	\centering
	\begin{tabular}{|x{\textwidth}|}
		\hline 
		\cellcolor{light_gray} \textbf{MAPEO DE RESULTADOS DEL ESTUDIANTE} \\ \hline
	\end{tabular}
\end{table}

\begin{table}[H]
	\centering
	\begin{tabular}{|x{2cm}|p{10cm}|x{2.1cm}|}
		\hline 
		\textbf{[A]} & Conocimientos en computación & Nivel 2 \\ \hline
		\textbf{[B]} & Análisis de problemas & Nivel 1\\ \hline
		\textbf{[C]} & Diseño y desarrollo de soluciones & Nivel 1\\ \hline
		\textbf{[D]} & Trabajo individual y en equipo & Nivel 1\\ \hline
		\textbf{[H]} & Uso de herramientas modernas & Nivel 2\\ \hline
	\end{tabular}
\end{table}

\begin{table}[H]
	\centering
	\begin{tabular}{|x{\textwidth}|}
		\hline 
		\cellcolor{light_gray} \textbf{ESCALA PARA MOSTRAR LOS NIVELES} \\ \hline
	\end{tabular}
\end{table}

\begin{table}[H]
	\centering
	\begin{tabular}{|p{\textwidth}|}
		\hline 
		Escala para mostrar el nivel de desarrollo: \\
		\\
		- = No se desarrolla\\
		0 = Conoce\\
		1 = Comprende\\
		2 = Aplica en un nivel intermedio\\
		3 = Logra el Resultado del Estudiante\\ \hline
		
	\end{tabular}
\end{table}
%-------------------------------------------------------------------------------
%-------------------------------------------------------------------------------
