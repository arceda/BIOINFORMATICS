\documentclass{article}
\usepackage[utf8]{inputenc}
\usepackage[top=3cm, bottom=3cm, outer=3cm, inner=3cm]{geometry}
\usepackage{graphicx}
\usepackage{url}
\usepackage{cite}
\usepackage{hyperref}
\usepackage{array}
\usepackage{float}
\usepackage{multicol}
\newcolumntype{x}[1]{>{\centering\arraybackslash\hspace{0pt}}p{#1}}

%%%%%%%%%%%%%%%%%%%%%%%%%%%%%%%%%%%%%%%%%%%%%%%%%%%%%%%%%%%%%%%%%%%%%%%%%%%%
%%%%%%%%%%%%%%%%%%%%%%%%%%%%%%%%%%%%%%%%%%%%%%%%%%%%%%%%%%%%%%%%%%%%%%%%%%%%
\newcommand{\csemail}{vmachacaa@unsa.edu.pe}
\newcommand{\csdocente}{MSc. Vicente Machaca Arceda}
\newcommand{\cscurso}{Bioinformática}
\newcommand{\csuniversidad}{Universidad Nacional de San Agustín de Arequipa}
\newcommand{\csescuela}{Escuela Profesional de Ciencia de la Computación}
\newcommand{\cspracnr}{03}
\newcommand{\cstema}{Dot plot}
%%%%%%%%%%%%%%%%%%%%%%%%%%%%%%%%%%%%%%%%%%%%%%%%%%%%%%%%%%%%%%%%%%%%%%%%%%%%
%%%%%%%%%%%%%%%%%%%%%%%%%%%%%%%%%%%%%%%%%%%%%%%%%%%%%%%%%%%%%%%%%%%%%%%%%%%%

\usepackage{fancyhdr}
\pagestyle{fancy}
\fancyhf{}
\setlength{\headheight}{30pt}
\renewcommand{\headrulewidth}{1pt}
\renewcommand{\footrulewidth}{1pt}
\fancyhead[L]{ \raisebox{0.1\height}{\includegraphics[width=3cm]{img/logo_unsa}} }
\fancyhead[C]{ \fontsize{7}{7}\selectfont	\csuniversidad \\ \csescuela \\ \textbf{\cscurso} \\ \raisebox{\height}{ } }
\fancyhead[R]{ \raisebox{0.1\height}{\includegraphics[width=1.5cm]{img/logo_epcc_unsa}} }
\fancyfoot[L]{MSc. Vicente Machaca}
\fancyfoot[C]{\cscurso}
\fancyfoot[R]{Página \thepage}


\usepackage{listings}
\usepackage{color}
\definecolor{dkgreen}{rgb}{0,0.6,0}
\definecolor{gray}{rgb}{0.5,0.5,0.5}
\definecolor{mauve}{rgb}{0.58,0,0.82}
\lstset{frame=tb,
	language=Python,
	aboveskip=3mm,
	belowskip=3mm,
	showstringspaces=false,
	columns=flexible,
	basicstyle={\small\ttfamily},
	numbers=none,
	numberstyle=\tiny\color{gray},
	keywordstyle=\color{blue},
	commentstyle=\color{dkgreen},
	stringstyle=\color{mauve},
	breaklines=true,
	breakatwhitespace=true,
	tabsize=3
}

\usepackage[english,spanish]{babel}
\AtBeginDocument{\selectlanguage{spanish}}
\renewcommand{\figurename}{Figura}
\renewcommand{\refname}{Referencias}
\renewcommand{\tablename}{Tabla}


% para la imagen de fondo
%\usepackage{eso-pic}
%\newcommand\BackgroundPic{%
%\put(0,0){%
%\parbox[b][\paperheight]{\paperwidth}{%
%\vfill
%\centering
%\includegraphics[width=\paperwidth,height=\paperheight]{../img/background4.png}%
%\vfill
%}}}


%\title{\textbf{Lista de proyectos}}
%\author{\csdocente}
%\date{\today}
		
		
\begin{document}
	
	\begin{center}	
	\fontsize{15}{15} \textbf{Práctica \cspracnr}
\end{center}
% image background
%%%%%%%%%%%%%%%%%%%%%%%%%%%%%%%%%%%%%%%%%%%%%%%%%%%%%%%%%%%%%%%%%%%%%%%%%%	
%%%%%%%%%%%%%%%%%%%%%%%%%%%%%%%%%%%%%%%%%%%%%%%%%%%%%%%%%%%%%%%%%%%%%%%%%%	
%\AddToShipoutPicture{\BackgroundPic}
%\AddToShipoutPicture*{\BackgroundPic} %solo laprimera página
%%%%%%%%%%%%%%%%%%%%%%%%%%%%%%%%%%%%%%%%%%%%%%%%%%%%%%%%%%%%%%%%%%%%%%%%%%	
%%%%%%%%%%%%%%%%%%%%%%%%%%%%%%%%%%%%%%%%%%%%%%%%%%%%%%%%%%%%%%%%%%%%%%%%%%	


	
	%\maketitle
	
	\begin{table}[h]
		\begin{tabular}{|x{4.7cm}|x{4.8cm}|x{4.8cm}|}
			\hline 
			\textbf{DOCENTE} & \textbf{CARRERA}  & \textbf{CURSO}   \\
			\hline 
			\csdocente & \csescuela & \cscurso    \\
			\hline 
		\end{tabular}
	\end{table}
	
	\begin{table}[h]
		\begin{tabular}{|x{4.7cm}|x{4.8cm}|x{4.8cm}|}
			\hline 
			\textbf{PRÁCTICA} & \textbf{TEMA}  & \textbf{DURACIÓN}   \\
			\hline 
			\cspracnr & \cstema & 2 horas   \\
			\hline 
		\end{tabular}
	\end{table}
	
	
	\section{Competencias del curso}
	\begin{itemize}
		\item Aplica las bases matemáticas y la teoría de la informática en algoritmos de Bioinformática.
		\item Analiza, diseña y propone soluciones frente a problemas bioinformáticos.
		\item Sabe cómo utilizar y conoce las bases computacionales de herramientas modernas de secuenciamiento,
		alineamiento, árboles filogenéticos y mapeo de genomas.
	\end{itemize}
	
	
	\section{Competencias de la práctica}
	\begin{itemize}
		\item Utilizar herramientas de Dot plot.
		\item Implementar el algoritmo Dot plot para alineamiento de secuencias.
	\end{itemize}
	
	\section{Equipos y materiales}
	\begin{itemize}
		\item Python
		\item Matplotlib 
		\item Numpy 
		\item BioPython
		\item Cuenta en Github
	\end{itemize}
	
	\section{Entregables}
	\begin{itemize}
		\item Se debe elaborar un informe en Latex donde se responda a cada ejercicio de la Sección \ref{sec:ejercicios}.
		\item En el informe se debe agregar un enlace al repositorio Github donde esta el código.
		\item En el informe se debe agregar el código fuente asi como capturas de pantalla de la ejecución y resultados del mismo.
	\end{itemize}
	
	
	
	
	\clearpage
	
	
	\section{Ejercicios}\label{sec:ejercicios}
	\begin{enumerate}
		\item Descarge dos secuencias de proteinas (las utilizadas en clases) y ejecute el siguiente código. Comente sus resultados.
		\begin{lstlisting}
		# dot_matrix.py
		from Bio import SeqIO
		
		sequences = SeqIO.parse("P21333.fasta", "fasta")
		for record in sequences:
		data1 = str(record.seq.upper()) # the fasta file just have one sequence 
		
		sequences = SeqIO.parse("Q8BTM8.fasta", "fasta")
		for record in sequences:
		data2 = str(record.seq.upper()) # the fasta file just have one sequence  
		
		print(data1)
		print(data2)
		\end{lstlisting}
		
		\item Ahora usted debe implementar un programa que genere un Dot matrix. Se recomienda utilizar Matplot para la gráfica, de igual manera no es necesario dibujar las lineas, basta con dibujar los puntos por cada coincidencia (no olvide incluir el \textit{windows size} y un \textit{threshold}).
		
		\item Descargue otras secuencias y genere el \textit{Dot matrix}, evalue sus resultados y comente.
	\end{enumerate}
	
	
	%\clearpage
	%\bibliographystyle{ieeetr}
	%\bibliography{../bibliography}
	
\end{document}
