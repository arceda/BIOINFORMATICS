\documentclass{article}
\usepackage[utf8]{inputenc}
\usepackage[top=3cm, bottom=3cm, outer=3cm, inner=3cm]{geometry}
\usepackage{graphicx}
\usepackage{url}
\usepackage{cite}
\usepackage{hyperref}
\usepackage{array}
\usepackage{multicol}
\newcolumntype{x}[1]{>{\centering\arraybackslash\hspace{0pt}}p{#1}}


%%%%%%%%%%%%%%%%%%%%%%%%%%%%%%%%%%%%%%%%%%%%%%%%%%%%%%%%%%%%%%%%%%%%%%%%%%%%
%%%%%%%%%%%%%%%%%%%%%%%%%%%%%%%%%%%%%%%%%%%%%%%%%%%%%%%%%%%%%%%%%%%%%%%%%%%%
\newcommand{\csemail}{vmachacaa@unsa.edu.pe}
\newcommand{\csdocente}{MSc. Vicente Machaca Arceda}
\newcommand{\cscurso}{Bioinformática}
\newcommand{\csuniversidad}{Universidad Nacional de San Agustín de Arequipa}
\newcommand{\csescuela}{Escuela Profesional de Ciencia de la Computación}
\newcommand{\cspracnr}{08}
\newcommand{\cstema}{BLAST}
%%%%%%%%%%%%%%%%%%%%%%%%%%%%%%%%%%%%%%%%%%%%%%%%%%%%%%%%%%%%%%%%%%%%%%%%%%%%
%%%%%%%%%%%%%%%%%%%%%%%%%%%%%%%%%%%%%%%%%%%%%%%%%%%%%%%%%%%%%%%%%%%%%%%%%%%%


\usepackage{fancyhdr}
\pagestyle{fancy}
\fancyhf{}
\setlength{\headheight}{30pt}
\renewcommand{\headrulewidth}{1pt}
\renewcommand{\footrulewidth}{1pt}
\fancyhead[L]{ \raisebox{0.1\height}{\includegraphics[width=3cm]{img/logo_unsa}} }
\fancyhead[C]{ \fontsize{7}{7}\selectfont	\csuniversidad \\ \csescuela \\ \textbf{\cscurso} \\ \raisebox{\height}{ } }
\fancyhead[R]{ \raisebox{0.1\height}{\includegraphics[width=1.5cm]{img/logo_epcc_unsa}} }
\fancyfoot[L]{MSc. Vicente Machaca}
\fancyfoot[C]{\cscurso}
\fancyfoot[R]{Página \thepage}

\usepackage{listings}
\usepackage{color}
\definecolor{lightgray}{rgb}{.9,.9,.9}
\definecolor{darkgray}{rgb}{.4,.4,.4}
\definecolor{purple}{rgb}{0.65, 0.12, 0.82}
\definecolor{forestgreen}{rgb}{0.12, 0.6, 0.32}
\definecolor{blue}{rgb}{0, 0,1}

\lstdefinelanguage{JavaScript}{
	keywords={typeof, new, true, false, catch, function, return, null, catch, switch, var, if, in, while, do, else, case, break, for, let, const},
	keywordstyle=\color{blue}\bfseries,
	ndkeywords={class, export, boolean, throw, implements, import, this},
	ndkeywordstyle=\color{blue}\bfseries,
	identifierstyle=\color{black},
	sensitive=false,
	comment=[l]{//},
	morecomment=[s]{/*}{*/},
	commentstyle=\color{forestgreen}\ttfamily,
	stringstyle=\color{red}\ttfamily,
	morestring=[b]',
	morestring=[b]"
}

\lstset{frame=tb,
	language=JavaScript,
	%backgroundcolor=\color{white},
	extendedchars=true,
	basicstyle=\footnotesize\ttfamily,
	showstringspaces=false,
	showspaces=false,
	numbers=none,
	numberstyle=\footnotesize,
	numbersep=9pt,
	tabsize=2,
	breaklines=true,
	showtabs=false,
	captionpos=b
}

\usepackage[english,spanish]{babel}
\AtBeginDocument{\selectlanguage{spanish}}
\renewcommand{\figurename}{Figura}
\renewcommand{\refname}{Referencias}
\renewcommand{\tablename}{Tabla}
		
		
		
\begin{document}
	
	% image background
	%%%%%%%%%%%%%%%%%%%%%%%%%%%%%%%%%%%%%%%%%%%%%%%%%%%%%%%%%%%%%%%%%%%%%%%%%%	
	%%%%%%%%%%%%%%%%%%%%%%%%%%%%%%%%%%%%%%%%%%%%%%%%%%%%%%%%%%%%%%%%%%%%%%%%%%	
	%\AddToShipoutPicture{\BackgroundPic}
	%\AddToShipoutPicture*{\BackgroundPic} %solo laprimera página
	%%%%%%%%%%%%%%%%%%%%%%%%%%%%%%%%%%%%%%%%%%%%%%%%%%%%%%%%%%%%%%%%%%%%%%%%%%	
	%%%%%%%%%%%%%%%%%%%%%%%%%%%%%%%%%%%%%%%%%%%%%%%%%%%%%%%%%%%%%%%%%%%%%%%%%%	
	
	\begin{center}	
		\fontsize{15}{15} \textbf{Práctica \cspracnr}
	\end{center}
	
	%\maketitle
	
	\begin{table}[h]
		\begin{tabular}{|x{4.7cm}|x{4.8cm}|x{4.8cm}|}
			\hline 
			\textbf{DOCENTE} & \textbf{CARRERA}  & \textbf{CURSO}   \\
			\hline 
			\csdocente & \csescuela & \cscurso    \\
			\hline 
		\end{tabular}
	\end{table}
	
	\begin{table}[h]
		\begin{tabular}{|x{4.7cm}|x{4.8cm}|x{4.8cm}|}
			\hline 
			\textbf{PRÁCTICA} & \textbf{TEMA}  & \textbf{DURACIÓN}   \\
			\hline 
			\cspracnr & \cstema & 3 horas   \\
			\hline 
		\end{tabular}
	\end{table}
	
	
	\section{Resultados del estudiante}
	\begin{itemize}
		\item (a) Conocimientos en computación
		\item (b) Análisis de problemas.
		\item (c) Diseño y desarrollo de soluciones.
		\item (d) Trabajo individual y en equipo.
		\item (h) Uso de herramientas modernas.
	\end{itemize}
	
	
	\section{Competencias de la práctica}
	\begin{itemize}
		\item Analiza el problema de alineamiento y propone una solución utilizando el algoritmo BLAST. 
		\item Trabaja en equipo para el desarrollo de soluciones.
	\end{itemize}
	
	\section{Equipos y materiales}
	\begin{itemize}
		\item Latex
		\item Python
		\item BioPython
	\end{itemize}
	
	\section{Entregables}
	\begin{itemize}
		\item Se debe elaborar un informe en Latex.
		\item El informe debe contener un enlace a github, pruebas y conclusiones.
	\end{itemize}
	
	
	
	
	\clearpage
	
	
	\section{Ejercicios}\label{sec:ejercicios}
		

	%\subsection{Pregunta 1}
	\begin{enumerate}		
		\item Construya una pequeña base de datos de ADN's o aminoacidos. Puede utilizar secuencias de \href{https://www.uniprot.org/}{UniProt}, \href{http://castor.bioinfo.uqam.ca/}{Castor} o las que usted desee.
		 
		 \item Implemente el algoritmo original de BLAST y evalue sus resultados en la base de datos construida anteriormente. 
	\end{enumerate}

\clearpage
\section{Rúbricas}

\begin{table}[hbt!]
	\setlength{\tabcolsep}{0.5em} % for the horizontal padding
	{\renewcommand{\arraystretch}{1.5}% for the vertical padding
		\begin{tabular}{|p{5cm}|x{3cm}|x{3cm}|x{3cm}|}
			\hline 
			\textbf{Rúbrica} & \textbf{Cumple}  & \textbf{Cumple con obs.}  & \textbf{No cumple} \\
			\hline 
			\textbf{Informe}: Desarrolla un informe, con un formato limpio y facil de leer. Además, utiliza la plantilla brindada por el docente.   & 2 & 1 & 0   \\ 
			\hline 
			\textbf{Implementación}: Implementa el algoritmo BLAST [b, c].   & 10 & 5 & 0   \\ \hline			
			
			\textbf{Calidad de las pruebas}: Realiza las pruebas en una base de datos mayor a 10 secuencias y utiliza secuencias reales con mas de 1000 bases [c].  & 6 & 3 & 0   \\ \hline
			
			\textbf{Presentación}: El alumno demuestra el trabajo en equipo (github) y dominio del tema [d]. & 2 & 1 & 0   \\ 			\hline 
			
			\textbf{Errores ortográficos}: Por cada error ortográfico, se descontará 1 punto.  & - & - & -   \\ \hline
			
		\end{tabular}
	}
\end{table}

	%\clearpage
	%\bibliographystyle{ieeetr}
	%\bibliography{../bibliography}
	
\end{document}
