\documentclass{article}
\usepackage[utf8]{inputenc}
\usepackage[top=2cm, bottom=2cm, outer=2cm, inner=2cm]{geometry}
\usepackage{graphicx}
\usepackage{url}
\usepackage{cite}
\usepackage{hyperref}
\usepackage{array}
\usepackage{multicol}
\newcolumntype{x}[1]{>{\centering\arraybackslash\hspace{0pt}}p{#1}}


%%%%%%%%%%%%%%%%%%%%%%%%%%%%%%%%%%%%%%%%%%%%%%%%%%%%%%%%%%%%%%%%%%%%%%%%%%%%
%%%%%%%%%%%%%%%%%%%%%%%%%%%%%%%%%%%%%%%%%%%%%%%%%%%%%%%%%%%%%%%%%%%%%%%%%%%%
\newcommand{\csemail}{vmachacaa@unsa.edu.pe}
\newcommand{\csdocente}{MSc. Vicente Machaca Arceda}
\newcommand{\cscurso}{Computación Molecular Biológica}
\newcommand{\csuniversidad}{Universidad Nacional de San Agustín de Arequipa}
\newcommand{\csescuela}{Escuela Profesional de Ciencia de la Computación}
\newcommand{\cspracnr}{05}
\newcommand{\cstema}{Alineamiento de Secuencias con Programación Dinámica}
%%%%%%%%%%%%%%%%%%%%%%%%%%%%%%%%%%%%%%%%%%%%%%%%%%%%%%%%%%%%%%%%%%%%%%%%%%%%
%%%%%%%%%%%%%%%%%%%%%%%%%%%%%%%%%%%%%%%%%%%%%%%%%%%%%%%%%%%%%%%%%%%%%%%%%%%%


\usepackage{fancyhdr}
\pagestyle{fancy}
\fancyhf{}
\rhead{\cscurso}
\lhead{\csdocente}
\rfoot{Página \thepage}

% Logos in first page
\fancypagestyle{plain}{%
	\renewcommand{\headrulewidth}{0pt}%
	\fancyhf{}%
	\fancyfoot[C]{\footnotesize Página \thepage\ }%
	\fancyhead[L]{ \raisebox{-0.2\height}{\includegraphics[height=13mm]{img/logo_unsa.jpg}}   }
	\fancyhead[R]{ \raisebox{-0.2\height}{\includegraphics[height=13mm]{img/logo_epcc_unsa.png}}  }
	\fancyhead[C]{  \fontsize{8}{8}\selectfont
		\csuniversidad \\  
		\textbf{\csescuela} \\ 
		Curso: \cscurso 
	}
	%\renewcommand{\headrulewidth}{0.5pt}% Default \headrulewidth is 0.4pt
	%\renewcommand{\footrulewidth}{0.4pt}% Default \footrulewidth is 0pt
	
}
\headheight 40pt              %% put this outside
\headsep 10pt 


% para el codigo fuente
\usepackage{listings}
\usepackage{color}
\definecolor{dkgreen}{rgb}{0,0.6,0}
\definecolor{gray}{rgb}{0.5,0.5,0.5}
\definecolor{mauve}{rgb}{0.58,0,0.82}
\lstset{frame=tb,
	language=Python,
	aboveskip=3mm,
	belowskip=3mm,
	showstringspaces=false,
	columns=flexible,
	basicstyle={\small\ttfamily},
	numbers=none,
	numberstyle=\tiny\color{gray},
	keywordstyle=\color{blue},
	commentstyle=\color{dkgreen},
	stringstyle=\color{mauve},
	breaklines=true,
	breakatwhitespace=true,
	tabsize=3
}

\usepackage[english,spanish]{babel}
\AtBeginDocument{\selectlanguage{spanish}}
\renewcommand{\figurename}{Figura}
\renewcommand{\refname}{Referencias}
\renewcommand{\tablename}{Tabla}


% para la imagen de fondo
\usepackage{eso-pic}
\newcommand\BackgroundPic{%
	\put(0,0){%
		\parbox[b][\paperheight]{\paperwidth}{%
			\vfill
			\centering
			\includegraphics[width=\paperwidth,height=\paperheight]{../img/background4.png}%
			\vfill
		}}}
		
		
		\title{\textbf{Práctica \cspracnr}}
		\author{\csdocente}
		\date{\today}
		
		
		
\begin{document}
	
	% image background
	%%%%%%%%%%%%%%%%%%%%%%%%%%%%%%%%%%%%%%%%%%%%%%%%%%%%%%%%%%%%%%%%%%%%%%%%%%	
	%%%%%%%%%%%%%%%%%%%%%%%%%%%%%%%%%%%%%%%%%%%%%%%%%%%%%%%%%%%%%%%%%%%%%%%%%%	
	%\AddToShipoutPicture{\BackgroundPic}
	%\AddToShipoutPicture*{\BackgroundPic} %solo laprimera página
	%%%%%%%%%%%%%%%%%%%%%%%%%%%%%%%%%%%%%%%%%%%%%%%%%%%%%%%%%%%%%%%%%%%%%%%%%%	
	%%%%%%%%%%%%%%%%%%%%%%%%%%%%%%%%%%%%%%%%%%%%%%%%%%%%%%%%%%%%%%%%%%%%%%%%%%	
	
	
	
	\maketitle
	
	\begin{table}[h]
		\begin{tabular}{|x{5cm}|x{6cm}|x{5cm}|}
			\hline 
			\textbf{DOCENTE} & \textbf{CARRERA}  & \textbf{CURSO}   \\
			\hline 
			\csdocente & \csescuela & \cscurso    \\
			\hline 
		\end{tabular}
	\end{table}
	
	\begin{table}[h]
		\begin{tabular}{|x{5cm}|x{6cm}|x{5cm}|}
			\hline 
			\textbf{PRÁCTICA} & \textbf{TEMA}  & \textbf{DURACIÓN}   \\
			\hline 
			\cspracnr & \cstema & 3 horas   \\
			\hline 
		\end{tabular}
	\end{table}
	
	
	\section{Competencias del curso}
	\begin{itemize}
		\item Aplica las bases matemáticas y la teoría de la informática en algoritmos de Bioinformática.
		\item Analiza, diseña y propone soluciones frente a problemas bioinformáticos.
		\item Sabe cómo utilizar y conoce las bases computacionales de herramientas modernas de secuenciamiento,
		alineamiento, árboles filogenéticos y mapeo de genomas.
	\end{itemize}
	
	
	\section{Competencias de la práctica}
	\begin{itemize}
		\item Aplica las bases matemáticas y la teoría de la informática en algoritmos de \cstema.
	\end{itemize}
	
	\section{Equipos y materiales}
	\begin{itemize}
		\item Latex
		\item Conección a internet 
		\item Python
		\item Matplotlib 
		\item Numpy 
		\item BioPython
		\item Cuenta en Github
	\end{itemize}
	
	\section{Entregables}
	\begin{itemize}
		%\item Se debe elaborar un informe donde se explique cada tema asignado.
		%\item Exposición de cada tema, cada exposición no debe durar mas de 10 min.
		\item Se debe elaborar un informe en Latex donde se responda a cada ejercicio de la Sección \ref{sec:ejercicios}.
		\item En el informe se debe agregar un enlace al repositorio Github donde esta el código.
		\item En el informe se debe agregar el código fuente asi como capturas de pantalla de la ejecución y resultados del mismo.
	\end{itemize}
	
	
	
	
	\clearpage
	
	
	\section{Ejercicios}\label{sec:ejercicios}
		

	%\subsection{Pregunta 1}
	\begin{enumerate}
		
		\item 
		
		 Implemente el algoritmo de alineamiento de secuencias utilizando programación dinámica (Needleman–Wunsch). Evalue sus resultados con las secuencias:
				\begin{itemize}
					\item $S_1$: AGC
					\item $S_2$: AAG
				\end{itemize}
			
		Utilice $gapOpen = gapEXTEND = -5$ y la siguiente matriz de sustitución:
		
		\begin{table}[hbt!]
			\centering
			\begin{tabular}{l|llll}
				\textbf{} & A  & C  & G  & T  \\
				\hline
				A         & 2  & -7 & -5 & -7 \\
				C         & -7 & 2  & -7 & -5 \\
				G         & -5 & -7 & 2  & -7 \\
				T         & -7 & -5 & -7 & 2 
			\end{tabular}
		\end{table}
	
		La salida debe incluir el \textit{score matrix} y todos los alineamientos posibles.
			
		\item 
		
		Utilice el algoritmo implentado anteriormente, pero esta vez con secuencies reales. Se recomienda utilizar las secuencias de proteinas de la Práctica 3. Utilice $gapOpen = gapEXTEND = -5$, $identicalMatch = 2$ y $mismatch = -2$
		
		\item 
		Evalue la pregunta anterior con otros valores de $gapOpen, gapEXTEND, identicalMatch$ y $mismatch$. Verifique si obtiene el mismo alineamiento y comente sus resultados.
		

	

	\end{enumerate}
	
	%\clearpage
	%\bibliographystyle{ieeetr}
	%\bibliography{../bibliography}
	
\end{document}
