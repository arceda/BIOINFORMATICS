\documentclass{article}
\usepackage[utf8]{inputenc}
\usepackage[top=2cm, bottom=2cm, outer=2cm, inner=2cm]{geometry}
\usepackage{graphicx}
\usepackage{url}
\usepackage{cite}
\usepackage{hyperref}
\usepackage{array}
\usepackage{multicol}
\newcolumntype{x}[1]{>{\centering\arraybackslash\hspace{0pt}}p{#1}}

\usepackage{fancyhdr}
\pagestyle{fancy}
\fancyhf{}
\rhead{Computación Molecular Biológica}
\lhead{MSc. Vicente Machaca Arceda}
%\rfoot{Página \thepage}
\rfoot{Página \thepage}

% Logos in first page
\fancypagestyle{plain}{%
	\renewcommand{\headrulewidth}{0pt}%
	\fancyhf{}%
	\fancyfoot[C]{\footnotesize Página \thepage\ }%
	\fancyhead[L]{ \raisebox{-0.2\height}{\includegraphics[height=13mm]{img/logo_unsa.jpg}}   }
	\fancyhead[R]{ \raisebox{-0.2\height}{\includegraphics[height=13mm]{img/logo_epcc_unsa.png}}  }
	\fancyhead[C]{  \fontsize{8}{8}\selectfont
		Universidad Nacional de San Agustín de Arequipa \\  
		\textbf{Escuela Profesional de Ciencia de la Computación} \\ 
		Curso: Computación Molecular Biológica 
	}
	%\renewcommand{\headrulewidth}{0.5pt}% Default \headrulewidth is 0.4pt
	%\renewcommand{\footrulewidth}{0.4pt}% Default \footrulewidth is 0pt
	
}
\headheight 40pt              %% put this outside
\headsep 10pt 


% para el codigo fuente
\usepackage{listings}
\usepackage{color}
\definecolor{dkgreen}{rgb}{0,0.6,0}
\definecolor{gray}{rgb}{0.5,0.5,0.5}
\definecolor{mauve}{rgb}{0.58,0,0.82}
\lstset{frame=tb,
	language=Python,
	aboveskip=3mm,
	belowskip=3mm,
	showstringspaces=false,
	columns=flexible,
	basicstyle={\small\ttfamily},
	numbers=none,
	numberstyle=\tiny\color{gray},
	keywordstyle=\color{blue},
	commentstyle=\color{dkgreen},
	stringstyle=\color{mauve},
	breaklines=true,
	breakatwhitespace=true,
	tabsize=3
}

\usepackage[english,french,spanish]{babel}
\AtBeginDocument{\selectlanguage{spanish}}
\renewcommand{\figurename}{Figura}
\renewcommand{\refname}{Referencias}
\renewcommand{\tablename}{Tabla}


% para la imagen de fondo
\usepackage{eso-pic}
\newcommand\BackgroundPic{%
	\put(0,0){%
		\parbox[b][\paperheight]{\paperwidth}{%
			\vfill
			\centering
			\includegraphics[width=\paperwidth,height=\paperheight]{../img/background4.png}%
			\vfill
		}}}
		
		
		\title{\textbf{Práctica 1}}
		\author{MSc. Vicente Machaca Arceda}
		\date{\today}
		
		
		
\begin{document}
	
	% image background
	%%%%%%%%%%%%%%%%%%%%%%%%%%%%%%%%%%%%%%%%%%%%%%%%%%%%%%%%%%%%%%%%%%%%%%%%%%	
	%%%%%%%%%%%%%%%%%%%%%%%%%%%%%%%%%%%%%%%%%%%%%%%%%%%%%%%%%%%%%%%%%%%%%%%%%%	
	%\AddToShipoutPicture{\BackgroundPic}
	%\AddToShipoutPicture*{\BackgroundPic} %solo laprimera página
	%%%%%%%%%%%%%%%%%%%%%%%%%%%%%%%%%%%%%%%%%%%%%%%%%%%%%%%%%%%%%%%%%%%%%%%%%%	
	%%%%%%%%%%%%%%%%%%%%%%%%%%%%%%%%%%%%%%%%%%%%%%%%%%%%%%%%%%%%%%%%%%%%%%%%%%	
	
	
	
	\maketitle
	
	\begin{table}[h]
		\begin{tabular}{|x{5cm}|x{6cm}|x{5cm}|}
			\hline 
			\textbf{DOCENTE} & \textbf{CARRERA}  & \textbf{CURSO}   \\
			\hline 
			MSc. Vicente Machaca Arceda & Escuela Profesional de Ciencias de la Computación & Computación molecular Biológica    \\
			\hline 
		\end{tabular}
	\end{table}
	
	\begin{table}[h]
		\begin{tabular}{|x{5cm}|x{6cm}|x{5cm}|}
			\hline 
			\textbf{PRÁCTICA} & \textbf{TEMA}  & \textbf{DURACIÓN}   \\
			\hline 
			01 & Thresholding & 3 horas   \\
			\hline 
		\end{tabular}
	\end{table}
	
	
	\section{Competencias del curso}
	\begin{itemize}
		\item Aplica las bases matemáticas y la teoría de la informática en algoritmos de Bioinformática.
		\item Analiza, diseña y propone soluciones frente a problemas bioinformáticos.
		\item Sabe cómo utilizar y conoce las bases computacionales de herramientas modernas de secuenciamiento,
		alineamiento, árboles filogenéticos y mapeo de genomas.
	\end{itemize}
	
	
	\section{Competencias de la práctica}
	\begin{itemize}
		\item Utilizar herramientas de Dot plot.
		\item Implementar el algoritmo Dot plot para alineamiento de secuencias.
	\end{itemize}
	
	\section{Equipos y materiales}
	\begin{itemize}
		\item Python
		\item Matplotlib 
		\item Numpy 
		\item BioPython
		\item Cuenta en Github
	\end{itemize}
	
	\section{Entregables}
	\begin{itemize}
		\item Se debe elaborar un informe en Latex donde se responda a cada ejercicio de la Sección \ref{sec:ejercicios}.
		\item En el informe se debe agregar un enlace al repositorio Github donde esta el código.
		\item En el informe se debe agregar el código fuente asi como capturas de pantalla de la ejecución y resultados del mismo.
	\end{itemize}
	
	
	
	
	\clearpage
	
	
	\section{Ejercicios}\label{sec:ejercicios}
	\begin{enumerate}
		\item El siguiente código lee los 2 archivos descargados anteriormente y muestra las secuencias por pantalla.
		\begin{lstlisting}
		# dot_matrix.py
		from Bio import SeqIO
		
		sequences = SeqIO.parse("P21333.fasta", "fasta")
		for record in sequences:
		data1 = str(record.seq.upper()) # the fasta file just have one sequence 
		
		sequences = SeqIO.parse("Q8BTM8.fasta", "fasta")
		for record in sequences:
		data2 = str(record.seq.upper()) # the fasta file just have one sequence  
		
		print(data1)
		print(data2)
		\end{lstlisting}
		
		\item Ahora usted debe implementar un programa que genere un Dot matrix. Se recomienda utilizar Matplot para la gráfica, de igual manera no es necesario dibujar las lineas, basta con dibujar los puntos por cada coincidencia.
		
		\item Descargue otras secuencias y genere el \textit{Dot matrix}, evalue sus resultados y comente.
	\end{enumerate}
	
	
	%\clearpage
	%\bibliographystyle{ieeetr}
	%\bibliography{../bibliography}
	
\end{document}
