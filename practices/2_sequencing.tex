\documentclass{article}
\usepackage[utf8]{inputenc}
\usepackage[top=2cm, bottom=2cm, outer=2cm, inner=2cm]{geometry}
\usepackage{graphicx}
\usepackage{url}
\usepackage{cite}
\usepackage{hyperref}
\usepackage{array}
\usepackage{multicol}
\newcolumntype{x}[1]{>{\centering\arraybackslash\hspace{0pt}}p{#1}}

%%%%%%%%%%%%%%%%%%%%%%%%%%%%%%%%%%%%%%%%%%%%%%%%%%%%%%%%%%%%%%%%%%%%%%%%%%%%
%%%%%%%%%%%%%%%%%%%%%%%%%%%%%%%%%%%%%%%%%%%%%%%%%%%%%%%%%%%%%%%%%%%%%%%%%%%%
\newcommand{\csemail}{vmachacaa@unsa.edu.pe}
\newcommand{\csdocente}{MSc. Vicente Machaca Arceda}
\newcommand{\cscurso}{Computación Molecular Biológica}
\newcommand{\csuniversidad}{Universidad Nacional de San Agustín de Arequipa}
\newcommand{\csescuela}{Escuela Profesional de Ciencia de la Computación}
\newcommand{\cspracnr}{02}
\newcommand{\cstema}{Next Generation Sequence}
%%%%%%%%%%%%%%%%%%%%%%%%%%%%%%%%%%%%%%%%%%%%%%%%%%%%%%%%%%%%%%%%%%%%%%%%%%%%
%%%%%%%%%%%%%%%%%%%%%%%%%%%%%%%%%%%%%%%%%%%%%%%%%%%%%%%%%%%%%%%%%%%%%%%%%%%%

\usepackage{fancyhdr}


\pagestyle{fancy}
\fancyhf{}
\rhead{\cscurso}
\lhead{\csdocente}
\rfoot{Página \thepage}

% Logos in first page
\fancypagestyle{plain}{%
	\renewcommand{\headrulewidth}{0pt}%
	\fancyhf{}%
	\fancyfoot[C]{\footnotesize Página \thepage\ }%
	\fancyhead[L]{ \raisebox{-0.2\height}{\includegraphics[height=13mm]{img/logo_unsa.jpg}}   }
	\fancyhead[R]{ \raisebox{-0.2\height}{\includegraphics[height=13mm]{img/logo_epcc_unsa.png}}  }
	\fancyhead[C]{  \fontsize{8}{8}\selectfont
		\csuniversidad \\  
		\textbf{\csescuela} \\ 
		Curso: \cscurso 
	}
	%\renewcommand{\headrulewidth}{0.5pt}% Default \headrulewidth is 0.4pt
	%\renewcommand{\footrulewidth}{0.4pt}% Default \footrulewidth is 0pt
	
}
\headheight 40pt              %% put this outside
\headsep 10pt 


% para el codigo fuente
\usepackage{listings}
\usepackage{color}
\definecolor{dkgreen}{rgb}{0,0.6,0}
\definecolor{gray}{rgb}{0.5,0.5,0.5}
\definecolor{mauve}{rgb}{0.58,0,0.82}
\lstset{frame=tb,
	language=Python,
	aboveskip=3mm,
	belowskip=3mm,
	showstringspaces=false,
	columns=flexible,
	basicstyle={\small\ttfamily},
	numbers=none,
	numberstyle=\tiny\color{gray},
	keywordstyle=\color{blue},
	commentstyle=\color{dkgreen},
	stringstyle=\color{mauve},
	breaklines=true,
	breakatwhitespace=true,
	tabsize=3
}

\usepackage[english,french,spanish]{babel}
\AtBeginDocument{\selectlanguage{spanish}}
\renewcommand{\figurename}{Figura}
\renewcommand{\refname}{Referencias}
\renewcommand{\tablename}{Tabla}


% para la imagen de fondo
\usepackage{eso-pic}
\newcommand\BackgroundPic{%
	\put(0,0){%
		\parbox[b][\paperheight]{\paperwidth}{%
			\vfill
			\centering
			\includegraphics[width=\paperwidth,height=\paperheight]{../img/background4.png}%
			\vfill
		}}}
		
		
		
		
		
\title{\textbf{Práctica 2}}
\author{\csdocente}
\date{\today}
		
		
		
\begin{document}
	
	% image background
	%%%%%%%%%%%%%%%%%%%%%%%%%%%%%%%%%%%%%%%%%%%%%%%%%%%%%%%%%%%%%%%%%%%%%%%%%%	
	%%%%%%%%%%%%%%%%%%%%%%%%%%%%%%%%%%%%%%%%%%%%%%%%%%%%%%%%%%%%%%%%%%%%%%%%%%	
	%\AddToShipoutPicture{\BackgroundPic}
	%\AddToShipoutPicture*{\BackgroundPic} %solo laprimera página
	%%%%%%%%%%%%%%%%%%%%%%%%%%%%%%%%%%%%%%%%%%%%%%%%%%%%%%%%%%%%%%%%%%%%%%%%%%	
	%%%%%%%%%%%%%%%%%%%%%%%%%%%%%%%%%%%%%%%%%%%%%%%%%%%%%%%%%%%%%%%%%%%%%%%%%%	
	
	
	
	\maketitle
	
	\begin{table}[h]
		\begin{tabular}{|x{5cm}|x{6cm}|x{5cm}|}
			\hline 
			\textbf{DOCENTE} & \textbf{CARRERA}  & \textbf{CURSO}   \\
			\hline 
			\csdocente & \csescuela & \cscurso    \\
			\hline 
		\end{tabular}
	\end{table}
	
	\begin{table}[h]
		\begin{tabular}{|x{5cm}|x{6cm}|x{5cm}|}
			\hline 
			\textbf{PRÁCTICA} & \textbf{TEMA}  & \textbf{DURACIÓN}   \\
			\hline 
			02 & Secuenciamiento de ADN & 3 horas   \\
			\hline 
		\end{tabular}
	\end{table}
	
	
	\section{Competencias del curso}
	\begin{itemize}
		\item Aplica las bases matemáticas y la teoría de la informática en algoritmos de Bioinformática.
		\item Analiza, diseña y propone soluciones frente a problemas bioinformáticos.
		\item Sabe cómo utilizar y conoce las bases computacionales de herramientas modernas de secuenciamiento,
		alineamiento, árboles filogenéticos y mapeo de genomas.
	\end{itemize}
	
	
	\section{Competencias de la práctica}
	\begin{itemize}
		\item Comprender los métodos utilizados en \textit{Next Generation Sequence} (NGS)
	\end{itemize}
	
	\section{Equipos y materiales}
	\begin{itemize}
		\item Editor de texto  Latex 
	\end{itemize}
	
	\section{Entregables}
	\begin{itemize}
		\item Se debe elaborar un informe y una presentación en Latex donde se desarrolle el trabajo solicitado.
		\item El informe se desarrollará en grupos de 4.
		\item El informe deberá estar correctamente citado utilizando las normas APA.
	\end{itemize}
		
	
	
	\clearpage
	
	
	\section{Desarrollo}\label{sec:ejercicios}
	\begin{enumerate}
		\item Escoga uno de los siguiente temas:
			\begin{itemize}
				\item Pyrosequencing (454 Life Sciences)
				\item Semiconductor sequencing (Ion Torrent).			
				\item Reversible chain-termination sequencing (Illumina).			
				\item Single-molecule sequencing (Pacific Biosciences and MinION)	
				\item RNA-seq.
				\item DNA microarray.
			\end{itemize}
		\item Desarrolle un informe con la siguiente estructura como minimo:
			\begin{itemize}
				\item Introducción
				\item Descripción del método.
					\begin{itemize}
						\item Pasos o etapas del método.
						\item Longitud de fragmentos leidos.
					\end{itemize}
				\item Ventajas y desventajas.			
				\item Conclusiones
				\item Referencias
			\end{itemize}
			
	\end{enumerate}
	
	
	%\clearpage
	%\bibliographystyle{ieeetr}
	%\bibliography{../bibliography}
	
\end{document}
