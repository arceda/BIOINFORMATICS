\documentclass{article}
\usepackage[utf8]{inputenc}
\usepackage[top=3cm, bottom=3cm, outer=3cm, inner=3cm]{geometry}
\usepackage{graphicx}
\usepackage{url}
\usepackage{cite}
\usepackage{hyperref}
\usepackage{array}
\usepackage{multicol}
\newcolumntype{x}[1]{>{\centering\arraybackslash\hspace{0pt}}p{#1}}


%%%%%%%%%%%%%%%%%%%%%%%%%%%%%%%%%%%%%%%%%%%%%%%%%%%%%%%%%%%%%%%%%%%%%%%%%%%%
%%%%%%%%%%%%%%%%%%%%%%%%%%%%%%%%%%%%%%%%%%%%%%%%%%%%%%%%%%%%%%%%%%%%%%%%%%%%
\newcommand{\csemail}{vmachacaa@unsa.edu.pe}
\newcommand{\csdocente}{MSc. Vicente Machaca Arceda}
\newcommand{\cscurso}{Bioinformática}
\newcommand{\csuniversidad}{Universidad Nacional de San Agustín de Arequipa}
\newcommand{\csescuela}{Escuela Profesional de Ciencia de la Computación}
\newcommand{\cspracnr}{09}
\newcommand{\cstema}{UPGMA}
%%%%%%%%%%%%%%%%%%%%%%%%%%%%%%%%%%%%%%%%%%%%%%%%%%%%%%%%%%%%%%%%%%%%%%%%%%%%
%%%%%%%%%%%%%%%%%%%%%%%%%%%%%%%%%%%%%%%%%%%%%%%%%%%%%%%%%%%%%%%%%%%%%%%%%%%%


\usepackage{fancyhdr}
\pagestyle{fancy}
\fancyhf{}
\setlength{\headheight}{30pt}
\renewcommand{\headrulewidth}{1pt}
\renewcommand{\footrulewidth}{1pt}
\fancyhead[L]{ \raisebox{0.1\height}{\includegraphics[width=3cm]{img/logo_unsa}} }
\fancyhead[C]{ \fontsize{7}{7}\selectfont	\csuniversidad \\ \csescuela \\ \textbf{\cscurso} \\ \raisebox{\height}{ } }
\fancyhead[R]{ \raisebox{0.1\height}{\includegraphics[width=1.5cm]{img/logo_epcc_unsa}} }
\fancyfoot[L]{MSc. Vicente Machaca}
\fancyfoot[C]{\cscurso}
\fancyfoot[R]{Página \thepage}


\usepackage[english,spanish]{babel}
\AtBeginDocument{\selectlanguage{spanish}}
\renewcommand{\figurename}{Figura}
\renewcommand{\refname}{Referencias}
\renewcommand{\tablename}{Tabla}
		
\usepackage{color}
\definecolor{lightgray}{rgb}{0.95, 0.95, 0.95}
\definecolor{gray}{rgb}{0.95, 0.95, 0.95}
\definecolor{darkgray}{rgb}{0.4, 0.4, 0.4}
%\definecolor{purple}{rgb}{0.65, 0.12, 0.82}
\definecolor{editorGray}{rgb}{0.95, 0.95, 0.95}
\definecolor{editorOcher}{rgb}{1, 0.5, 0} % #FF7F00 -> rgb(239, 169, 0)
\definecolor{editorGreen}{rgb}{0, 0.5, 0} % #007C00 -> rgb(0, 124, 0)
\definecolor{orange}{rgb}{1,0.45,0.13}		
\definecolor{olive}{rgb}{0.17,0.59,0.20}
\definecolor{brown}{rgb}{0.69,0.31,0.31}
\definecolor{purple}{rgb}{0.38,0.18,0.81}
\definecolor{lightblue}{rgb}{0.1,0.57,0.7}
\definecolor{lightred}{rgb}{1,0.4,0.5}
\definecolor{dkgreen}{rgb}{0,0.6,0}
\usepackage{upquote}
\usepackage{listings}
% CSS
\lstdefinelanguage{CSS}{
	keywords={color,background-image:,margin,padding,font,weight,display,position,top,left,right,bottom,list,style,border,size,white,space,min,width, transition:, transform:, transition-property, transition-duration, transition-timing-function},	
	sensitive=true,
	morecomment=[l]{//},
	morecomment=[s]{/*}{*/},
	morestring=[b]',
	morestring=[b]",
	alsoletter={:},
	alsodigit={-}
}

% JavaScript
\lstdefinelanguage{JavaScript}{
	keywords={typeof, new, true, false, catch, function, return, null, catch, switch, var, if, in, while, do, else, case, break, for, let, const},
	keywordstyle=\color{purple}\bfseries,
	ndkeywords={class, export, boolean, throw, implements, import, this},
	ndkeywordstyle=\color{purple}\bfseries,
	identifierstyle=\color{black},
	sensitive=false,
	comment=[l]{//},
	morecomment=[s]{/*}{*/},
	commentstyle=\color{blue}\ttfamily,
	stringstyle=\color{red}\ttfamily,
	morestring=[b]',
	morestring=[b]"
}

\lstdefinelanguage{HTML5}{
	language=html,
	sensitive=true,	
	alsoletter={<>=-},	
	morecomment=[s]{<!-}{-->},
	tag=[s],
	otherkeywords={
		% General
		>,
		% Standard tags
		<!DOCTYPE,
		</html, <html, <head, <title, </title, <style, </style, <link, </head, <meta, />,
		% body
		</body, <body,
		% Divs
		</div, <div, </div>, 
		% Paragraphs
		</p, <p, </p>,
		% scripts
		</script, <script,
		% More tags...
		<template, /template>, <canvas, /canvas>, <svg, <rect, <animateTransform, </rect>, </svg>, <video, <source, <iframe, </iframe>, </video>, <image, </image>, <header, </header, <article, </article
	},
	ndkeywords={
		% General
		=,
		% HTML attributes
		charset=, src=, id=, width=, height=, style=, type=, rel=, href=,
		% SVG attributes
		fill=, attributeName=, begin=, dur=, from=, to=, poster=, controls=, x=, y=, repeatCount=, xlink:href=,
		% properties
		margin:, padding:, background-image:, border:, top:, left:, position:, width:, height:, margin-top:, margin-bottom:, font-size:, line-height:,
		% CSS3 properties
		transform:, -moz-transform:, -webkit-transform:,
		animation:, -webkit-animation:,
		transition:,  transition-duration:, transition-property:, transition-timing-function:,
	}
}

\lstdefinestyle{javascript}{
	frame=tbrl,
	language=JavaScript,
	aboveskip=3mm,
	belowskip=3mm,
	showstringspaces=false,
	columns=flexible,
	basicstyle={\small\ttfamily},
	numbers=none,
	numberstyle=\tiny\color{gray},
	keywordstyle=\color{blue}\bfseries,
	commentstyle=\color{dkgreen}\ttfamily,
	stringstyle=\color{editorOcher}\ttfamily,
	breaklines=true,
	numbers=left,
	breakatwhitespace=true,
	tabsize=2
}

\lstdefinestyle{html}{
	frame=tbrl,
	language=HTML5,
	alsolanguage=JavaScript,
	aboveskip=3mm,
	belowskip=3mm,
	showstringspaces=false,
	columns=flexible,
	basicstyle={\small\ttfamily},
	numbers=none,
	numberstyle=\tiny\color{gray},
	keywordstyle=\color{blue}\bfseries,
	commentstyle=\color{dkgreen}\ttfamily,
	stringstyle=\color{editorOcher}\ttfamily,
	breaklines=true,
	numbers=left,
	breakatwhitespace=true,
	tabsize=2
}

\lstdefinestyle{python}{
	frame=tbrl,
	language=python,
	aboveskip=3mm,
	belowskip=3mm,
	numberfirstline=true,	
	numbers=left,
	showstringspaces=false,
	columns=flexible,
	basicstyle={\small\ttfamily},
	numbers=none,
	numberstyle=\tiny\color{gray},
	keywordstyle=\color{blue}\bfseries,
	commentstyle=\color{dkgreen}\ttfamily,
	stringstyle=\color{editorOcher}\ttfamily,
	breaklines=true,
	numbers=left,
	breakatwhitespace=true,
	tabsize=2
}

\lstdefinestyle{htmlcssjs} {%
	% General design
	%  backgroundcolor=\color{editorGray},
	basicstyle={\footnotesize\ttfamily},   
	frame=tb,
	% line-numbers
	xleftmargin={0.75cm},
	numbers=left,
	stepnumber=1,
	firstnumber=1,
	numberfirstline=true,	
	% Code design
	identifierstyle=\color{black},
	keywordstyle=\color{blue}\bfseries,
	ndkeywordstyle=\color{editorGreen}\bfseries,
	stringstyle=\color{editorOcher}\ttfamily,
	commentstyle=\color{brown}\ttfamily,
	% Code
	language=HTML5,
	alsolanguage=JavaScript,
	alsodigit={.:;},	
	tabsize=2,
	showtabs=false,
	showspaces=false,
	showstringspaces=false,
	extendedchars=true,
	breaklines=true,
	% German umlauts
	literate=%
	{Ö}{{\"O}}1
	{Ä}{{\"A}}1
	{Ü}{{\"U}}1
	{ß}{{\ss}}1
	{ü}{{\"u}}1
	{ä}{{\"a}}1
	{ö}{{\"o}}1
}
%
\lstdefinestyle{py} {%
	language=python,
	literate=%
	*{0}{{{\color{lightred}0}}}1
	{1}{{{\color{lightred}1}}}1
	{2}{{{\color{lightred}2}}}1
	{3}{{{\color{lightred}3}}}1
	{4}{{{\color{lightred}4}}}1
	{5}{{{\color{lightred}5}}}1
	{6}{{{\color{lightred}6}}}1
	{7}{{{\color{lightred}7}}}1
	{8}{{{\color{lightred}8}}}1
	{9}{{{\color{lightred}9}}}1,
	basicstyle=\footnotesize\ttfamily, % Standardschrift
	numbers=left,               % Ort der Zeilennummern
	%numberstyle=\tiny,          % Stil der Zeilennummern
	%stepnumber=2,               % Abstand zwischen den Zeilennummern
	numbersep=5pt,              % Abstand der Nummern zum Text
	tabsize=4,                  % Groesse von Tabs
	extendedchars=true,         %
	breaklines=true,            % Zeilen werden Umgebrochen
	keywordstyle=\color{blue}\bfseries,
	frame=b,
	commentstyle=\color{brown}\itshape,
	stringstyle=\color{editorOcher}\ttfamily, % Farbe der String
	showspaces=false,           % Leerzeichen anzeigen ?
	showtabs=false,             % Tabs anzeigen ?
	xleftmargin=17pt,
	framexleftmargin=17pt,
	framexrightmargin=5pt,
	framexbottommargin=4pt,
	%backgroundcolor=\color{lightgray},
	showstringspaces=false,      % Leerzeichen in Strings anzeigen ?
}%
%

	
		
\begin{document}
	
	% image background
	%%%%%%%%%%%%%%%%%%%%%%%%%%%%%%%%%%%%%%%%%%%%%%%%%%%%%%%%%%%%%%%%%%%%%%%%%%	
	%%%%%%%%%%%%%%%%%%%%%%%%%%%%%%%%%%%%%%%%%%%%%%%%%%%%%%%%%%%%%%%%%%%%%%%%%%	
	%\AddToShipoutPicture{\BackgroundPic}
	%\AddToShipoutPicture*{\BackgroundPic} %solo laprimera página
	%%%%%%%%%%%%%%%%%%%%%%%%%%%%%%%%%%%%%%%%%%%%%%%%%%%%%%%%%%%%%%%%%%%%%%%%%%	
	%%%%%%%%%%%%%%%%%%%%%%%%%%%%%%%%%%%%%%%%%%%%%%%%%%%%%%%%%%%%%%%%%%%%%%%%%%	
	
	\begin{center}	
		\fontsize{15}{15} \textbf{Práctica \cspracnr}
	\end{center}
	
	%\maketitle
	
	\begin{table}[h]
		\begin{tabular}{|x{4.7cm}|x{4.8cm}|x{4.8cm}|}
			\hline 
			\textbf{DOCENTE} & \textbf{CARRERA}  & \textbf{CURSO}   \\
			\hline 
			\csdocente & \csescuela & \cscurso    \\
			\hline 
		\end{tabular}
	\end{table}
	
	\begin{table}[h]
		\begin{tabular}{|x{4.7cm}|x{4.8cm}|x{4.8cm}|}
			\hline 
			\textbf{PRÁCTICA} & \textbf{TEMA}  & \textbf{DURACIÓN}   \\
			\hline 
			\cspracnr & \cstema & 3 horas   \\
			\hline 
		\end{tabular}
	\end{table}
	
	
	\section{Resultados del estudiante}
	\begin{itemize}
		\item (a) Conocimientos en computación
		\item (b) Análisis de problemas.
		\item (c) Diseño y desarrollo de soluciones.
		\item (d) Trabajo individual y en equipo.
		\item (h) Uso de herramientas modernas.
	\end{itemize}
	
	
	\section{Competencias de la práctica}
	\begin{itemize}
		\item Implementa el algoritmo UPGMA para la construcción árboles filogenéticos. 
		\item Utiliza herramientas como BioPython y ETE para la implementación y análisis de árboles filogenéticos.
	\end{itemize}
	
	\section{Equipos y materiales}
	\begin{itemize}
		\item Latex
		\item Python
		\item BioPython
	\end{itemize}
	
	\section{Entregables}
	\begin{itemize}
		\item Se debe elaborar un informe en Latex.
		\item El informe debe contener un enlace a github, pruebas y conclusiones.
	\end{itemize}
	
	
	
	
	\clearpage
	
	
	\section{Ejercicios}\label{sec:ejercicios}
		

	%\subsection{Pregunta 1}
	\begin{enumerate}		
		\item Instale las librerías BioPython, \href{http://scikit-bio.org/}{scikit-bio} y \href{http://etetoolkit.org/}{ETE} para el análisis de árboles filogenéticos. Además describa las librerías, en un parrafo de no mas de 5 lineas.
		 
		 \item Evalue el código a continuación y describa su funcionamiento.
		 
		 \begin{lstlisting}[style=python]
from ete3 import PhyloTree, TreeStyle
from skbio import DistanceMatrix
from skbio.tree import nj

data = [[0, 8, 4, 6],
		[8, 0, 8, 8],
		[4, 8, 0, 6],         
		[6, 8, 6, 0]]
ids = list('abcd')

dm = DistanceMatrix(data, ids)
tree = nj(dm) # build a tree using neigbors joining algorithm
print(tree.ascii_art())

newick_str = nj(dm, result_constructor=str) # return newick format
print(newick_str)

t = PhyloTree(newick_str) # plot three using ETE
t.show()
		 \end{lstlisting}	
		 
		 
		\item Evalue el código a continuación y describa su funcionamiento. 
		
		\begin{lstlisting}[style=python]
from ete3 import PhyloTree, TreeStyle
from skbio import DistanceMatrix
from skbio.tree import nj

fasta_txt = """
>seqA
MAEIPDETIQQFMALT---HNIAVQYLSEFGDLNEALNSYYASQTDDIKDRREEAH
>seqB
MAEIPDATIQQFMALTNVSHNIAVQY--EFGDLNEALNSYYAYQTDDQKDRREEAH
>seqC
MAEIPDATIQ---ALTNVSHNIAVQYLSEFGDLNEALNSYYASQTDDQPDRREEAH
>seqD
MAEAPDETIQQFMALTNVSHNIAVQYLSEFGDLNEAL--------------REEAH
"""

# Load a tree and link it to an alignment.
t = PhyloTree("(((seqA,seqB),seqC),seqD);")
t.link_to_alignment(alignment=fasta_txt, alg_format="fasta")
t.show()
		\end{lstlisting}
		
		\item Implemente el algoritmo UPGMA. Este debe tomar como entrada una matriz de distancias y debe retornar el árbol filogenético en formato \textit{newick}. Luego utilice la librería \textit{ete3} para visualizar el árbol. Puede tomar como entrada la matriz de la pregunta 2.
		 
	\end{enumerate}

\clearpage
\section{Rúbricas}

\begin{table}[hbt!]
	\setlength{\tabcolsep}{0.5em} % for the horizontal padding
	{\renewcommand{\arraystretch}{1.5}% for the vertical padding
		\begin{tabular}{|p{5cm}|x{3cm}|x{3cm}|x{3cm}|}
			\hline 
			\textbf{Rúbrica} & \textbf{Cumple}  & \textbf{Cumple con obs.}  & \textbf{No cumple} \\
			\hline 
			%\textbf{Informe}: Desarrolla un informe, con un formato limpio y facil de leer. Además, utiliza la plantilla brindada por el docente.   & 2 & 1 & 0   \\ 

			\textbf{Implementación}: Implementa de manera eficiente y correcta el algoritmo UPGMA [b, c].   & 10 & 5 & 0   \\ \hline			
			
			\textbf{Uso de herramientas}: Aprende el correcto uso de las librerías scikit-bio y ETE durante el desarrollo de las preguntas 1, 2 y 3 [h].  & 5 & 2.5 & 0   \\ \hline
			
			\textbf{Informe y presentación}: Desarrolla un informe en Latex. Además, el alumno demuestra el trabajo en equipo (github) y dominio del tema durante la exposición [d]. & 5 & 2.5 & 0   \\ 			\hline 
			
			\textbf{Errores ortográficos}: Por cada error ortográfico, se descontará 1 punto.  & - & - & -   \\ \hline
			
		\end{tabular}
	}
\end{table}

	%\clearpage
	%\bibliographystyle{ieeetr}
	%\bibliography{../bibliography}
	
\end{document}
