\documentclass{article}
\usepackage[utf8]{inputenc}
\usepackage[top=2cm, bottom=2cm, outer=2cm, inner=2cm]{geometry}
\usepackage{graphicx}
\usepackage{url}
\usepackage{cite}
\usepackage{hyperref}
\usepackage{array}
\usepackage{multicol}
\newcolumntype{x}[1]{>{\centering\arraybackslash\hspace{0pt}}p{#1}}


%%%%%%%%%%%%%%%%%%%%%%%%%%%%%%%%%%%%%%%%%%%%%%%%%%%%%%%%%%%%%%%%%%%%%%%%%%%%
%%%%%%%%%%%%%%%%%%%%%%%%%%%%%%%%%%%%%%%%%%%%%%%%%%%%%%%%%%%%%%%%%%%%%%%%%%%%
\newcommand{\csemail}{vmachacaa@unsa.edu.pe}
\newcommand{\csdocente}{MSc. Vicente Machaca Arceda}
\newcommand{\cscurso}{Computación Molecular Biológica}
\newcommand{\csuniversidad}{Universidad Nacional de San Agustín de Arequipa}
\newcommand{\csescuela}{Escuela Profesional de Ciencia de la Computación}
\newcommand{\cspracnr}{04}
\newcommand{\cstema}{Lecturas}
%%%%%%%%%%%%%%%%%%%%%%%%%%%%%%%%%%%%%%%%%%%%%%%%%%%%%%%%%%%%%%%%%%%%%%%%%%%%
%%%%%%%%%%%%%%%%%%%%%%%%%%%%%%%%%%%%%%%%%%%%%%%%%%%%%%%%%%%%%%%%%%%%%%%%%%%%


\usepackage{fancyhdr}
\pagestyle{fancy}
\fancyhf{}
\rhead{\cscurso}
\lhead{\csdocente}
\rfoot{Página \thepage}

% Logos in first page
\fancypagestyle{plain}{%
	\renewcommand{\headrulewidth}{0pt}%
	\fancyhf{}%
	\fancyfoot[C]{\footnotesize Página \thepage\ }%
	\fancyhead[L]{ \raisebox{-0.2\height}{\includegraphics[height=13mm]{img/logo_unsa.jpg}}   }
	\fancyhead[R]{ \raisebox{-0.2\height}{\includegraphics[height=13mm]{img/logo_epcc_unsa.png}}  }
	\fancyhead[C]{  \fontsize{8}{8}\selectfont
		\csuniversidad \\  
		\textbf{\csescuela} \\ 
		Curso: \cscurso 
	}
	%\renewcommand{\headrulewidth}{0.5pt}% Default \headrulewidth is 0.4pt
	%\renewcommand{\footrulewidth}{0.4pt}% Default \footrulewidth is 0pt
	
}
\headheight 40pt              %% put this outside
\headsep 10pt 


% para el codigo fuente
\usepackage{listings}
\usepackage{color}
\definecolor{dkgreen}{rgb}{0,0.6,0}
\definecolor{gray}{rgb}{0.5,0.5,0.5}
\definecolor{mauve}{rgb}{0.58,0,0.82}
\lstset{frame=tb,
	language=Python,
	aboveskip=3mm,
	belowskip=3mm,
	showstringspaces=false,
	columns=flexible,
	basicstyle={\small\ttfamily},
	numbers=none,
	numberstyle=\tiny\color{gray},
	keywordstyle=\color{blue},
	commentstyle=\color{dkgreen},
	stringstyle=\color{mauve},
	breaklines=true,
	breakatwhitespace=true,
	tabsize=3
}

\usepackage[english,spanish]{babel}
\AtBeginDocument{\selectlanguage{spanish}}
\renewcommand{\figurename}{Figura}
\renewcommand{\refname}{Referencias}
\renewcommand{\tablename}{Tabla}


% para la imagen de fondo
\usepackage{eso-pic}
\newcommand\BackgroundPic{%
	\put(0,0){%
		\parbox[b][\paperheight]{\paperwidth}{%
			\vfill
			\centering
			\includegraphics[width=\paperwidth,height=\paperheight]{../img/background4.png}%
			\vfill
		}}}
		
		
		\title{\textbf{Práctica \cspracnr}}
		\author{\csdocente}
		\date{\today}
		
		
		
\begin{document}
	
	% image background
	%%%%%%%%%%%%%%%%%%%%%%%%%%%%%%%%%%%%%%%%%%%%%%%%%%%%%%%%%%%%%%%%%%%%%%%%%%	
	%%%%%%%%%%%%%%%%%%%%%%%%%%%%%%%%%%%%%%%%%%%%%%%%%%%%%%%%%%%%%%%%%%%%%%%%%%	
	%\AddToShipoutPicture{\BackgroundPic}
	%\AddToShipoutPicture*{\BackgroundPic} %solo laprimera página
	%%%%%%%%%%%%%%%%%%%%%%%%%%%%%%%%%%%%%%%%%%%%%%%%%%%%%%%%%%%%%%%%%%%%%%%%%%	
	%%%%%%%%%%%%%%%%%%%%%%%%%%%%%%%%%%%%%%%%%%%%%%%%%%%%%%%%%%%%%%%%%%%%%%%%%%	
	
	
	
	\maketitle
	
	\begin{table}[h]
		\begin{tabular}{|x{5cm}|x{6cm}|x{5cm}|}
			\hline 
			\textbf{DOCENTE} & \textbf{CARRERA}  & \textbf{CURSO}   \\
			\hline 
			\csdocente & \csescuela & \cscurso    \\
			\hline 
		\end{tabular}
	\end{table}
	
	\begin{table}[h]
		\begin{tabular}{|x{5cm}|x{6cm}|x{5cm}|}
			\hline 
			\textbf{PRÁCTICA} & \textbf{TEMA}  & \textbf{DURACIÓN}   \\
			\hline 
			\cspracnr & \cstema & 3 horas   \\
			\hline 
		\end{tabular}
	\end{table}
	
	
	\section{Competencias del curso}
	\begin{itemize}
		\item Aplica las bases matemáticas y la teoría de la informática en algoritmos de Bioinformática.
		\item Analiza, diseña y propone soluciones frente a problemas bioinformáticos.
		\item Sabe cómo utilizar y conoce las bases computacionales de herramientas modernas de secuenciamiento,
		alineamiento, árboles filogenéticos y mapeo de genomas.
	\end{itemize}
	
	
	\section{Competencias de la práctica}
	\begin{itemize}
		\item Comprender algunos tópicos en mutaciones, historia de la bioinformática y \textit{Virology}.
	\end{itemize}
	
	\section{Equipos y materiales}
	\begin{itemize}
		\item Latex
		\item Conección a internet 
		%\item Python
		%\item Matplotlib 
		%\item Numpy 
		%\item BioPython
		%\item Cuenta en Github
	\end{itemize}
	
	\section{Entregables}
	\begin{itemize}
		\item Se debe elaborar un informe donde se explique cada tema asignado.
		\item Exposición de cada tema, cada exposición no debe durar mas de 10 min.
		%\item Se debe elaborar un informe en Latex donde se responda a cada ejercicio de la Sección \ref{sec:ejercicios}.
		%\item En el informe se debe agregar un enlace al repositorio Github donde esta el código.
		%\item En el informe se debe agregar el código fuente asi como capturas de pantalla de la ejecución y resultados del mismo.
	\end{itemize}
	
	
	
	
	\clearpage
	
	
	%\section{Ejercicios}\label{sec:ejercicios}
	\section{Temas}\label{sec:ejercicios}
	
	A continuación se presentan los temas, cada grupo escogerá un tema y revisará el material de ayuda como mínimo.
	
	\begin{enumerate}
		\item Historia de la Bioinformática. Se debe explicar cada hito inmportante de la Bioinformática, sustentar cada evento con las publicaciones en revistas cientificas. Material de ayuda:
			\begin{itemize}
				\item \href{https://www.coursera.org/learn/bioinformatics-pku/lecture/0i4EF/history-of-bioinformatics}{Video}
				\item \href{https://journals.plos.org/ploscompbiol/article?id=10.1371/journal.pcbi.1000809}{The Roots of Bioinformatics}
			\end{itemize}
		\item Mutaciones. Material de ayuda:
			\begin{itemize}
				\item \href{https://www.nature.com/scitable/topicpage/dna-replication-and-causes-of-mutation-409/}{DNA Replication and Causes of Mutation}
				\item \href{https://www.nature.com/scitable/topicpage/genetic-mutation-441/}{Genetic Mutation}
				\item \href{https://www.ncbi.nlm.nih.gov/books/NBK21578/}{Mutations: Types and Causes}
			\end{itemize}
		
		\item Replicación de virus. Material de ayuda:
		\begin{itemize}
			\item \href{https://reader.elsevier.com/reader/sd/pii/B9780123751560000047?token=9B13E6DC096F4D4A969692D3BE8C9F7F7A9EB515D7CEC08023BB74B04F1BDD0C115766A9665D276C5990CAA8124564E6}{Virus Replication}
			\item \href{https://courses.lumenlearning.com/boundless-microbiology/chapter/dna-viruses-in-eukaryotes/}{DNA Viruses in Eukaryotes}
			
		\end{itemize}
	
		\item COVID-19. Material de ayuda:
		\begin{itemize}
			\item \href{https://drive.google.com/file/d/19nQoOZv8gpxrDPQtlZHTMKygRZD2f15s/view?usp=sharing}{The Human Coronavirus Disease COVID-19}
			\item \href{https://www.asm.org/COVID/COVID-19-Research-Registry/Basic-Virology}{COVID-19 Research regestry}
			
		\end{itemize}
		
	\end{enumerate}
	
	
	%\clearpage
	%\bibliographystyle{ieeetr}
	%\bibliography{../bibliography}
	
\end{document}
